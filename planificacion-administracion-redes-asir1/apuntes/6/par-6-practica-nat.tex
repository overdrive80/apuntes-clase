%%%%%%%%%%%%%%%%%%%%%%%%%%%%%%%%%%%%%%%%%
% Programming/Coding Assignment
% LaTeX Template
%
% This template has been downloaded from:
% http://www.latextemplates.com
%
% Original author:
% Ted Pavlic (http://www.tedpavlic.com)
%
% Note:
% The \lipsum[#] commands throughout this template generate dummy text
% to fill the template out. These commands should all be removed when 
% writing assignment content.
%
% This template uses a Perl script as an example snippet of code, most other
% languages are also usable. Configure them in the "CODE INCLUSION 
% CONFIGURATION" section.
%
%%%%%%%%%%%%%%%%%%%%%%%%%%%%%%%%%%%%%%%%%

%----------------------------------------------------------------------------------------
%	PACKAGES AND OTHER DOCUMENT CONFIGURATIONS
%----------------------------------------------------------------------------------------

\documentclass[a4paper]{article}
\usepackage[utf8]{inputenc}
\usepackage{listingsutf8}
\usepackage[spanish]{babel}




\usepackage{fancyhdr} % Required for custom headers
\usepackage{lastpage} % Required to determine the last page for the footer
\usepackage{extramarks} % Required for headers and footers
\usepackage[usenames,dvipsnames]{color} % Required for custom colors
\usepackage{graphicx} % Required to insert images
\usepackage{listings} % Required for insertion of code
\usepackage{courier} % Required for the courier font
\usepackage{lipsum} % Used for inserting dummy 'Lorem ipsum' text into the template
\usepackage{svg}
\usepackage{attachfile}
\usepackage{currfile}
\usepackage{multicol}
\usepackage{alltt}
\usepackage{framed}
\usepackage{caption}
\usepackage{seqsplit}

\hypersetup{
colorlinks,
citecolor=black,
filecolor=black,
linkcolor=black,
urlcolor=blue
}


% Margins
\topmargin=-0.45in
\evensidemargin=0in
\oddsidemargin=0in
\textwidth=6.5in
\textheight=9.8in
\headsep=0.25in

\linespread{1.1} % Line spacing
\setlength{\parindent}{1.5em}
\setlength{\parskip}{1em}


% Set up the header and footer
\pagestyle{fancy}
%\lhead{\hmwkAuthorName} % Top left header
\lhead{\hmwkClass}
\rhead{\hmwkTitle} % Top right header
\chead{} % Top center head
\lfoot{\lastxmark} % Bottom left footer
\cfoot{} % Bottom center footer
\rfoot{ \thepage\ / \protect\pageref{LastPage}} % Bottom right footer
\renewcommand\headrulewidth{0.4pt} % Size of the header rule
\renewcommand\footrulewidth{0.4pt} % Size of the footer rule



%----------------------------------------------------------------------------------------
%	CODE INCLUSION CONFIGURATION
%----------------------------------------------------------------------------------------
\renewcommand{\ttdefault}{pcr}
\definecolor{MyDarkGreen}{rgb}{0.0,0.4,0.0} % This is the color used for comments
\lstloadlanguages{Perl} % Load Perl syntax for listings, for a list of other languages supported see: ftp://ftp.tex.ac.uk/tex-archive/macros/latex/contrib/listings/listings.pdf
\lstset{
language=sh, % Use Perl in this example
frame=single, % Single frame around code
basicstyle=\small\ttfamily, % Use small true type font
keywordstyle=[1]\small\color{Blue}\bf, % Perl functions bold and blue
keywordstyle=[2]\small\color{Purple}, % Perl function arguments purple
keywordstyle=[3]\small\color{Blue}\underbar, % Custom functions underlined and blue
identifierstyle=, % Nothing special about identifiers                                         
commentstyle=\small\color{MyDarkGreen}, % Comments small dark green courier font
stringstyle=\small\color{Purple}, % Strings are purple
showstringspaces=false, % Don't put marks in string spaces
tabsize=5, % 5 spaces per tab
morecomment=[l][\small\color{Blue}]{...}, % Line continuation (...) like blue comment
numbers=none, % Line numbers on left
firstnumber=1, % Line numbers start with line 1
numberstyle=\tiny\color{Blue}, % Line numbers are blue and small
stepnumber=0 % Line numbers go in steps of 5
}

% Creates a new command to include a perl script, the first parameter is the filename of the script (without .pl), the second parameter is the caption
\newcommand{\perlscript}[2]{
\begin{itemize}
\item[]\lstinputlisting[caption=#2,label=#1]{#1.pl}
\end{itemize}
}

%----------------------------------------------------------------------------------------
%	DOCUMENT STRUCTURE COMMANDS
%	Skip this unless you know what you're doing
%----------------------------------------------------------------------------------------

% Header and footer for when a page split occurs within a problem environment
\newcommand{\enterProblemHeader}[1]{
%\nobreak\extramarks{#1}{#1 continued on next page\ldots}\nobreak
%\nobreak\extramarks{#1 (continued)}{#1 continued on next page\ldots}\nobreak
}

% Header and footer for when a page split occurs between problem environments
\newcommand{\exitProblemHeader}[1]{
%\nobreak\extramarks{#1 (continued)}{#1 continued on next page\ldots}\nobreak
%\nobreak\extramarks{#1}{}\nobreak
}

\setcounter{secnumdepth}{0} % Removes default section numbers
\newcounter{homeworkProblemCounter} % Creates a counter to keep track of the number of problems

\newcommand{\homeworkProblemName}{}
\newenvironment{homeworkProblem}[1][]{ % Makes a new environment called homeworkProblem which takes 1 argument (custom name) but the default is "Problem #"
\stepcounter{homeworkProblemCounter} % Increase counter for number of problems
\renewcommand{\homeworkProblemName}{Ejercicio \arabic{homeworkProblemCounter} #1} % Assign \homeworkProblemName the name of the problem
\section{\homeworkProblemName} % Make a section in the document with the custom problem count
\enterProblemHeader{\homeworkProblemName} % Header and footer within the environment
}{
\exitProblemHeader{\homeworkProblemName} % Header and footer after the environment
%\clearpage
}


\newcommand{\problemAnswer}[1]{ % Defines the problem answer command with the content as the only argument
\noindent\framebox[\columnwidth][c]{\begin{minipage}{0.98\columnwidth}#1\end{minipage}} % Makes the box around the problem answer and puts the content inside
}

\newcommand{\homeworkSectionName}{}
\newenvironment{homeworkSection}[1]{ % New environment for sections within homework problems, takes 1 argument - the name of the section
\renewcommand{\homeworkSectionName}{#1} % Assign \homeworkSectionName to the name of the section from the environment argument
\subsection{\homeworkSectionName} % Make a subsection with the custom name of the subsection
\enterProblemHeader{\homeworkProblemName\ [\homeworkSectionName]} % Header and footer within the environment
}{
\enterProblemHeader{\homeworkProblemName} % Header and footer after the environment
}

%----------------------------------------------------------------------------------------
%	NAME AND CLASS SECTION
%----------------------------------------------------------------------------------------

\newcommand{\hmwkTitle}{Redefine el hmwkTitle} % Assignment title
\newcommand{\hmwkDueDate}{asdfadsf} % Due date
\newcommand{\hmwkClass}{Redefine hmwkClass} % Course/class
\newcommand{\hmwkClassTime}{} % Class/lecture time
\newcommand{\hmwkClassInstructor}{} % Teacher/lecturer
\newcommand{\hmwkAuthorName}{Álvaro González Sotillo} % Your name

%----------------------------------------------------------------------------------------
%	TITLE PAGE
%----------------------------------------------------------------------------------------

\title{
\vspace{2in}
\textmd{\textbf{\hmwkClass:\ \hmwkTitle}}\\
\vspace{0.1in}\large{\textit{\hmwkClassInstructor\ \hmwkClassTime}}
\vspace{3in}
}

\author{\textbf{\hmwkAuthorName}}
\date{} % Insert date here if you want it to appear below your name


%----------------------------------------------------------------------------------------

\usepackage{fancybox}
\newcommand{\codigo}[1]{\texttt{#1}}


% CUADRITO
\newsavebox{\fmboxx}
\newenvironment{cuadrito}[1][14cm]
{\noindent \begin{center} \begin{lrbox}{\fmboxx}\noindent\begin{minipage}{#1}}
{\end{minipage}\end{lrbox}\noindent\shadowbox{\usebox{\fmboxx}} \end{center}}





\newenvironment{entradasalida}[2][14cm]
{
  \newcommand{\elnombredelafiguradeentradasalida}{#2}
  \begin{figure}[h]
    \begin{cuadrito}[#1]
      \begin{scriptsize}
\begin{alltt}
}
{%
\end{alltt}%
      \end{scriptsize}%
    \end{cuadrito}%
    \caption{\elnombredelafiguradeentradasalida}
  \end{figure}
}


\newenvironment{entradasalidacols}[2][14cm]
{
\newcommand{\elnombredelafiguradeentradasalida}{#2}
%\begin{wrapfigure}{}{0.1}
\begin{cuadrito}[#1]
\begin{scriptsize}
\begin{alltt}
}
{
\end{alltt}
\end{scriptsize}
\end{cuadrito}
\captionof{figure}{\elnombredelafiguradeentradasalida}
%\end{wrapfigure}
}


\usepackage{pgffor}
\newcommand{\ficheroautoref}[0]{%
  \foreach \ficherotex in {../../../common/plantilla-ejercicio.tex,../../common/plantilla-ejercicio.tex,../common/plantilla-ejercicio.tex,../apuntes/common/plantilla-ejercicio.tex}{

    \IfFileExists{\ficherotex}%
    {%
      \textattachfile[mimetype=text/plain,%
      description={La plantilla},%
      subject={La plantilla}]%
      {\ficherotex}%
      {}%
      % RECUPERO ESPACIO VERTICAL PERDIDO
      \vspace{-6em}%
    }%
    {}%
  }
  

  \textattachfile[mimetype=text/plain,
  description={El fichero TEX original para crear este documento, no sea que se nos pierda},
  subject={El fichero TEX original para crear este documento, no sea que se nos pierda}]
  {\currfilename}{}

}

\newcommand{\entradausuario}[1]{\textit{\textbf{#1}}}

\newcommand{\enlace}[2]{\textcolor{blue}{{\href{#1}{#2}}}}

\newcommand{\adjuntarfichero}[3]{
\textattachfile[mimetype=text/plain,
color={0 0 0},
description={#3},
subject={#1}]
{#1}
{\textcolor{blue}{\codigo{#2}}}
}

\newcommand{\adjuntardoc}[2]{%
\textattachfile[mimetype=text/plain,%
color={0 0 0},%
description={#1},%
subject={#1}]%
{#1}%
{\textcolor{blue}{#2}}%
}%


\newcommand{\plantilladeclase}[2]{
\adjuntarfichero{#1.java}{#2}{Plantilla para la clase #2}
}

% https://en.wikibooks.org/wiki/LaTeX/Source_Code_Listings#Encoding_issue
\lstset{literate=
  {á}{{\'a}}1 {é}{{\'e}}1 {í}{{\'i}}1 {ó}{{\'o}}1 {ú}{{\'u}}1
  {Á}{{\'A}}1 {É}{{\'E}}1 {Í}{{\'I}}1 {Ó}{{\'O}}1 {Ú}{{\'U}}1
  {à}{{\`a}}1 {è}{{\`e}}1 {ì}{{\`i}}1 {ò}{{\`o}}1 {ù}{{\`u}}1
  {À}{{\`A}}1 {È}{{\'E}}1 {Ì}{{\`I}}1 {Ò}{{\`O}}1 {Ù}{{\`U}}1
  {ä}{{\"a}}1 {ë}{{\"e}}1 {ï}{{\"i}}1 {ö}{{\"o}}1 {ü}{{\"u}}1
  {Ä}{{\"A}}1 {Ë}{{\"E}}1 {Ï}{{\"I}}1 {Ö}{{\"O}}1 {Ü}{{\"U}}1
  {â}{{\^a}}1 {ê}{{\^e}}1 {î}{{\^i}}1 {ô}{{\^o}}1 {û}{{\^u}}1
  {Â}{{\^A}}1 {Ê}{{\^E}}1 {Î}{{\^I}}1 {Ô}{{\^O}}1 {Û}{{\^U}}1
  {œ}{{\oe}}1 {Œ}{{\OE}}1 {æ}{{\ae}}1 {Æ}{{\AE}}1 {ß}{{\ss}}1
  {ű}{{\H{u}}}1 {Ű}{{\H{U}}}1 {ő}{{\H{o}}}1 {Ő}{{\H{O}}}1
  {ç}{{\c c}}1 {Ç}{{\c C}}1 {ø}{{\o}}1 {å}{{\r a}}1 {Å}{{\r A}}1
  {€}{{\euro}}1 {£}{{\pounds}}1 {«}{{\guillemotleft}}1
  {»}{{\guillemotright}}1 {ñ}{{\~n}}1 {Ñ}{{\~N}}1 {¿}{{?`}}1
}

\lstset{
  inputencoding=utf8,
  captionpos=b,
  frame=single,
  basicstyle=\small\ttfamily,
  showstringspaces=false,
  numbers=none,
  numbers=left,
  xleftmargin=2em,
  breaklines=true,
  postbreak=\mbox{\textcolor{red}{$\hookrightarrow$}\space}
}

\renewcommand{\lstlistingname}{Listado}
\captionsetup[lstlisting]{font={footnotesize},labelfont=bf,position=bottom}
\captionsetup[figure]{font={footnotesize},labelfont=bf}
\lstnewenvironment{listadojava}[2]
{
  \lstset{language=Java,caption={#1},label={#2}}
  \noindent\minipage{\linewidth}%
}
{\endminipage}

% LISTADO SHELL
\lstnewenvironment{listadoshell}[2]
{
  \lstset{language=bash,caption={#1},label={#2}}
  \noindent\minipage{\linewidth}%
}
{\endminipage}

% LISTADO TXT
\lstnewenvironment{listadotxt}[2]
{
  \lstset{caption={#1},label={#2}, keywords={}}
  \noindent\minipage{\linewidth}%
}
{\endminipage}

% LISTADO SQL
\lstnewenvironment{listadosql}[2]%
{%
  %(el 1 )caption es #1\\%
  %(el 2) label es #2\\%
  \lstset{language=sql,caption={#1},label={#2}}%
  \noindent\minipage{\linewidth}%
}
{\endminipage}
  






\newcommand{\graficosvg}[3][14cm]{
\begin{figure}[htbp]
\centering
\textattachfile{#2.svg}{
\color{black}
\includesvg[width=#1]{#2}
}
\caption{#3}
\end{figure}
}


\newcommand{\graficosvguml}[3][7cm]{
  \texttt{\graficosvg[#1]{#2}{#3}}
}


\newcommand{\primerapagina}{
\newpage
\ficheroautoref
\tableofcontents
\newpage
}

% CAJAS
\usepackage[skins]{tcolorbox}
\newtcolorbox{Aviso}[1][Aviso]{
  enhanced,
  colback=gray!5!white,
  colframe=gray!75!black,fonttitle=\bfseries,
  colbacktitle=gray!85!black,
  attach boxed title to top left={yshift=-2mm,xshift=2mm},
  title=#1
}


\newcommand{\StudentData}{
  \begin{cuadrito}[1\textwidth]
    \vspace{0.3cm}
    \large{
      \textbf{Apellidos:} \hrulefill \\
      \textbf{Nombre:} \hrulefill \\
      \textbf{Fecha:} \hrulefill \hspace{1cm} \textbf{Grupo:} \hrulefill \\
    }
    %\vspace{0.2cm}
  \end{cuadrito}
}

% TEXTO EN MONOESPACIO PERO QUE SE PARTE EN LINEAS
\usepackage{seqsplit}
\newcommand{\tecnico}[1]{\texttt{\seqsplit{#1}}}

% REEMPLAZAR TEXTO, NO LO USO
\def\replace#1#2#3{%
 \def\tmp##1#2{##1#3\tmp}%
   \tmp#1\stopreplace#2\stopreplace}
\def\stopreplace#1\stopreplace{}

\usepackage{eurosym}


\renewcommand{\hmwkClass}{Planificación y Administración de Redes}
\renewcommand{\hmwkTitle}{Práctica de conexiones TCP}




\begin{document}

% \maketitle

% ----------------------------------------------------------------------------------------
%	TABLE OF CONTENTS
% ----------------------------------------------------------------------------------------

% \setcounter{tocdepth}{1} % Uncomment this line if you don't want subsections listed in the ToC

\primerapagina

\setlength{\parindent}{0em}
\setlength{\parskip}{1em}


\section{Objetivo de la práctica}
Los objetivos  de la práctica son:
\begin{itemize}
\item Familiarizarse con los conceptos de puerto y conexión
\item Implementar NAT
\item Exponer servicios a través de un NAT
\end{itemize}


La última versión de esta práctica está disponible en \enlace{https://alvarogonzalezsotillo.github.io/apuntes-clase/planificacion-administracion-redes-asir1/apuntes/6/par-6-practica-nat.pdf}{este enlace}.


\section{Preparación del entorno}
\begin{itemize}
\item  \textbf{A}: Máquina virtual Linux (recomendado Debian)
  \begin{itemize}
  \item No necesita mucha memoria (256 MBytes)
    
  \item Tendrá una tarjeta de red \textit{bridged} con la dirección \texttt{192.168.255.N/24} (siendo \texttt{N} tu número de ordenador apuntada en  \url{http://bit.ly/2sji9gQ}.
  \item Tendrá una tarjeta de red \textit{bridged} con una IP en la red del aula, para la comunicación con Internet, en modo DHCP.
  \item Tendrá una tarjeta de red \textit{internal}, con una dirección privada clase B.

  \end{itemize}



\item \textbf{B}: Máquina virtual Windows, versión \textit{profesional}, \textit{ultimate} o equivalente.
  \begin{itemize}
  \item  No necesita mucha memoria (256 MBytes)
  \item Tendrá una tarjeta de red \textit{internal} , con una dirección privada clase B
  \end{itemize}
\item \textbf{C}: Máquina virtual Linux
  \begin{itemize}
  \item No necesita mucha memoria (256 MBytes)
  \item Tendrá una tarjeta de red \textit{internal}, con una dirección privada clase B
  \end{itemize}
  
\item  La máquina \textbf{A}, \textbf{B} y \textbf{C} deben poder conectarse a nivel IP por sus tarjetas \textit{internal}. \textbf{A} es el router por defecto de \textbf{B} y \textbf{C}.
\end{itemize}



\begin{homeworkProblem}[: Activa el NAT en la máquina A (2 puntos)]
  \begin{itemize}
  \item Activa el flag de enrutamiento de Linux en \textbf{A}
  \item Desactiva NAT. Recuerda volver a desactivarlo y comenzar de cero cada vez que realices una prueba.
    \begin{comment}
      iptables -t nat –F
    \end{comment}
  \item Activa NAT. Puedes usar, entre otras, las instrucciones de \enlace{http://albertomolina.wordpress.com/2009/01/09/nat-con-iptables/}{albertomolina}, apartado \textit{Source NAT}
  \item Comprueba que las máquinas \textbf{B} y \textbf{C} pueden acceder a internet, con \textbf{pathping} o similar.
  \end{itemize}
\end{homeworkProblem}

\begin{homeworkProblem}[: Conéctate a B por medio de escritorio remoto, desde tu PC físico (1 punto)]
  \begin{itemize}
    
  \item Busca qué puerto utiliza el protocolo RDP y ábrelo (\enlace{http://albertomolina.wordpress.com/2009/01/09/nat-con-iptables/}{albertomolina}, apartado prerouting)
  \item Activa el escritorio remoto en \textbf{B} (Propiedades del sistema $\rightarrow$ Remoto $\rightarrow$ Permitir escritorio remoto)
    
  \item Habilita algún usuario para que pueda conectarse por medio de escritorio remoto
  \item Conéctate desde tu PC físico con la orden \texttt{mstsc} en  Windows (o \texttt{rdesktop}/\texttt{remmina} si el ordenador físico tiene Linux)
  \end{itemize}

\end{homeworkProblem}


\begin{homeworkProblem}[: Conéctate a B desde el equipo B de otro compañero (1 punto)]
  Necesitarás repetir los pasos anteriores, pero para otra tarjeta de red (la de dirección \texttt{192.168.255.N}).
\end{homeworkProblem}


\begin{homeworkProblem}[: Convierte a B en un servidor de ficheros (carpetas compartidas) ]
  \begin{itemize}
  \item Abre los puertos necesarios en \textbf{A} para que pueda accederse a \textbf{B} como servidor de ficheros (\enlace{http://www.chicagotech.net/netforums/viewtopic.php?t=5067}{ayuda}).
  \item Debe poder accederse desde cualquier red externa.
  \item  Los puertos TCP se indican con el flag \texttt{-p tcp}. Los puertos UDP se indican con el flag \texttt{-p udp}.
    
  \item Comparte alguna carpeta en B. Recuerda que en Windows
    \begin{itemize}
    \item Es mejor desactivar “Utilizar el uso compartido simple de archivos”
    \item El usuario que se utiliza para acceder remotamente debe tener contraseña
    \end{itemize}

  \end{itemize}

\end{homeworkProblem}


\begin{homeworkProblem}[: Haz que A y C puedan ser administrados mediante SSH desde la red del aula ]
  \begin{itemize}
  \item Instala openssh-server en los dos ordenadores, si no está ya instalado.
  \item Abre los puertos necesarios para llegar a \textbf{C} por SSH desde la red del aula.
    
  \item Debe ser posible, desde la red del aula y la \texttt{192.168.255.N/24}, conectarse simultáneamente a \textbf{A} y \textbf{C} por SSH.
  \end{itemize}
\end{homeworkProblem}

\begin{homeworkProblem}[: Comprueba el NAT con \texttt{tcpdump} ]
  \begin{itemize}
  \item Abre dos consolas en la máquina \textbf{A}. Utiliza \texttt{tcpdump} para ver en una consola los paquetes ICMP enviados y recibidos (\enlace{https://forum.ivorde.com/tcpdump-how-to-to-capture-only-icmp-ping-echo-requests-t15191.html}{forum.ivorde.com}) por la tarjeta conectada a internet. En la otra consola, visualiza los paquetes ICMP de la tarjeta de red interna.
    \begin{itemize}
    \item También puede usarse \texttt{WireShark}
    \end{itemize}
  \item Ejecuta un \texttt{ping} a internet desde la máquina \textbf{B}.
  \item Comprueba que la máquina \textbf{A}, además de enrutar, está cambiado las direcciones IP de origen y destino.
  \end{itemize}
\end{homeworkProblem}


\section{Normas de entrega}
\begin{itemize}
\item Avisa al profesor para que compruebe el funcionamiento de cada ejercicio 
\item Se entregará una memoria con el formato acostumbrado. No es necesario explicar cada paso (en modo tutorial), solo demostrar que se han realizado las tareas.
\item La memoria incluirá

  \begin{itemize}
  \item Un esquema de la red 
  \item Las órdenes realizadas en el servidor Linux para conseguir realizar el NAT correctamente.
  \item Pantallazos que demuestren que se han realizado las tareas pedidas
  \item
    \begin{small}
      Nota: Las órdenes que manejan NAT en Linux deben comenzar siempre desde la primera. Se recomienda hacer un script SH para las órdenes.
    \end{small}
    
  \end{itemize}

\end{itemize}


\begin{comment}
#!/bin/bash

INTERNET=eth0
ALUMNOS=eth2
INTERNA=eth1

SERVIDORWINDOWS=172.16.0.1
LINUXINTERNO=172.16.0.2


# ACTIVA ENRUTAMIENTO
echo "1" > /proc/sys/net/ipv4/ip_forward


# LIMPIAR TABLAS
iptables -P INPUT ACCEPT
iptables -P FORWARD ACCEPT
iptables -P OUTPUT ACCEPT
iptables -t nat -F
iptables -t mangle -F
iptables -F
iptables -X

# NAT POR TARJETA CON INTERNET enp0s3
iptables -t nat -A POSTROUTING -o $INTERNET -j MASQUERADE

# NAT POR TARJETA ENTRE ALUMNOS
iptables -t nat -A POSTROUTING -o $ALUMNOS -j MASQUERADE

# ABRIR 3389 A SERVIDORWINDOWS
iptables -t nat -A PREROUTING -p tcp -i $INTERNET --dport 3389 -j DNAT --to $SERVIDORWINDOWS:3389
iptables -t nat -A PREROUTING -p tcp -i $ALUMNOS --dport 3389 -j DNAT --to $SERVIDORWINDOWS:3389

# ABRIR 222 A LINUXINTERNO
iptables -t nat -A PREROUTING -p tcp -i $INTERNET --dport 222 -j DNAT --to $LINUXINTERNO:22
iptables -t nat -A PREROUTING -p tcp -i $ALUMNOS --dport 222 -j DNAT --to $LINUXINTERNO:22


# CARPETAS COMPARTIDAS SERVIDOR WINDOWS
TCPPORTS="139 445"
UDPPORTS=""
for PORT in $TCPPORTS
do
    iptables -t nat -A PREROUTING -p tcp --dport $PORT -i $INTERNET -j DNAT --to $SERVIDORWINDOWS:$PORT
    iptables -t nat -A PREROUTING -p tcp --dport $PORT -i $ALUMNOS -j DNAT --to $SERVIDORWINDOWS:$PORT
done
for PORT in $UDPPORTS
do
    iptables -t nat -A PREROUTING -p udp --dport $PORT -i $INTERNET -j DNAT --to $SERVIDORWINDOWS:$PORT
    iptables -t nat -A PREROUTING -p udp --dport $PORT -i $ALUMNOS -j DNAT --to $SERVIDORWINDOWS:$PORT
done


# COMO QUEDAN LAS TABLAS
iptables-save

# VIGILAR EL PING PARA VER EL NAT
tcpdump -v -v -n -i $INTERNA icmp&
tcpdump -v -v -n -i $INTERNET icmp&


\end{comment}

\end{document}
