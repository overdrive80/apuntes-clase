%%%%%%%%%%%%%%%%%%%%%%%%%%%%%%%%%%%%%%%%%
% Programming/Coding Assignment
% LaTeX Template
%
% This template has been downloaded from:
% http://www.latextemplates.com
%
% Original author:
% Ted Pavlic (http://www.tedpavlic.com)
%
% Note:
% The \lipsum[#] commands throughout this template generate dummy text
% to fill the template out. These commands should all be removed when 
% writing assignment content.
%
% This template uses a Perl script as an example snippet of code, most other
% languages are also usable. Configure them in the "CODE INCLUSION 
% CONFIGURATION" section.
%
%%%%%%%%%%%%%%%%%%%%%%%%%%%%%%%%%%%%%%%%%

%----------------------------------------------------------------------------------------
%	PACKAGES AND OTHER DOCUMENT CONFIGURATIONS
%----------------------------------------------------------------------------------------

\documentclass[a4paper]{article}
\usepackage[utf8]{inputenc}
\usepackage{listingsutf8}
\usepackage[spanish]{babel}




\usepackage{fancyhdr} % Required for custom headers
\usepackage{lastpage} % Required to determine the last page for the footer
\usepackage{extramarks} % Required for headers and footers
\usepackage[usenames,dvipsnames]{color} % Required for custom colors
\usepackage{graphicx} % Required to insert images
\usepackage{listings} % Required for insertion of code
\usepackage{courier} % Required for the courier font
\usepackage{lipsum} % Used for inserting dummy 'Lorem ipsum' text into the template
\usepackage{svg}
\usepackage{attachfile}
\usepackage{currfile}
\usepackage{multicol}
\usepackage{alltt}
\usepackage{framed}
\usepackage{caption}
\usepackage{seqsplit}

\hypersetup{
colorlinks,
citecolor=black,
filecolor=black,
linkcolor=black,
urlcolor=blue
}


% Margins
\topmargin=-0.45in
\evensidemargin=0in
\oddsidemargin=0in
\textwidth=6.5in
\textheight=9.8in
\headsep=0.25in

\linespread{1.1} % Line spacing
\setlength{\parindent}{1.5em}
\setlength{\parskip}{1em}


% Set up the header and footer
\pagestyle{fancy}
%\lhead{\hmwkAuthorName} % Top left header
\lhead{\hmwkClass}
\rhead{\hmwkTitle} % Top right header
\chead{} % Top center head
\lfoot{\lastxmark} % Bottom left footer
\cfoot{} % Bottom center footer
\rfoot{ \thepage\ / \protect\pageref{LastPage}} % Bottom right footer
\renewcommand\headrulewidth{0.4pt} % Size of the header rule
\renewcommand\footrulewidth{0.4pt} % Size of the footer rule



%----------------------------------------------------------------------------------------
%	CODE INCLUSION CONFIGURATION
%----------------------------------------------------------------------------------------
\renewcommand{\ttdefault}{pcr}
\definecolor{MyDarkGreen}{rgb}{0.0,0.4,0.0} % This is the color used for comments
\lstloadlanguages{Perl} % Load Perl syntax for listings, for a list of other languages supported see: ftp://ftp.tex.ac.uk/tex-archive/macros/latex/contrib/listings/listings.pdf
\lstset{
language=sh, % Use Perl in this example
frame=single, % Single frame around code
basicstyle=\small\ttfamily, % Use small true type font
keywordstyle=[1]\small\color{Blue}\bf, % Perl functions bold and blue
keywordstyle=[2]\small\color{Purple}, % Perl function arguments purple
keywordstyle=[3]\small\color{Blue}\underbar, % Custom functions underlined and blue
identifierstyle=, % Nothing special about identifiers                                         
commentstyle=\small\color{MyDarkGreen}, % Comments small dark green courier font
stringstyle=\small\color{Purple}, % Strings are purple
showstringspaces=false, % Don't put marks in string spaces
tabsize=5, % 5 spaces per tab
morecomment=[l][\small\color{Blue}]{...}, % Line continuation (...) like blue comment
numbers=none, % Line numbers on left
firstnumber=1, % Line numbers start with line 1
numberstyle=\tiny\color{Blue}, % Line numbers are blue and small
stepnumber=0 % Line numbers go in steps of 5
}

% Creates a new command to include a perl script, the first parameter is the filename of the script (without .pl), the second parameter is the caption
\newcommand{\perlscript}[2]{
\begin{itemize}
\item[]\lstinputlisting[caption=#2,label=#1]{#1.pl}
\end{itemize}
}

%----------------------------------------------------------------------------------------
%	DOCUMENT STRUCTURE COMMANDS
%	Skip this unless you know what you're doing
%----------------------------------------------------------------------------------------

% Header and footer for when a page split occurs within a problem environment
\newcommand{\enterProblemHeader}[1]{
%\nobreak\extramarks{#1}{#1 continued on next page\ldots}\nobreak
%\nobreak\extramarks{#1 (continued)}{#1 continued on next page\ldots}\nobreak
}

% Header and footer for when a page split occurs between problem environments
\newcommand{\exitProblemHeader}[1]{
%\nobreak\extramarks{#1 (continued)}{#1 continued on next page\ldots}\nobreak
%\nobreak\extramarks{#1}{}\nobreak
}

\setcounter{secnumdepth}{0} % Removes default section numbers
\newcounter{homeworkProblemCounter} % Creates a counter to keep track of the number of problems

\newcommand{\homeworkProblemName}{}
\newenvironment{homeworkProblem}[1][]{ % Makes a new environment called homeworkProblem which takes 1 argument (custom name) but the default is "Problem #"
\stepcounter{homeworkProblemCounter} % Increase counter for number of problems
\renewcommand{\homeworkProblemName}{Ejercicio \arabic{homeworkProblemCounter} #1} % Assign \homeworkProblemName the name of the problem
\section{\homeworkProblemName} % Make a section in the document with the custom problem count
\enterProblemHeader{\homeworkProblemName} % Header and footer within the environment
}{
\exitProblemHeader{\homeworkProblemName} % Header and footer after the environment
%\clearpage
}


\newcommand{\problemAnswer}[1]{ % Defines the problem answer command with the content as the only argument
\noindent\framebox[\columnwidth][c]{\begin{minipage}{0.98\columnwidth}#1\end{minipage}} % Makes the box around the problem answer and puts the content inside
}

\newcommand{\homeworkSectionName}{}
\newenvironment{homeworkSection}[1]{ % New environment for sections within homework problems, takes 1 argument - the name of the section
\renewcommand{\homeworkSectionName}{#1} % Assign \homeworkSectionName to the name of the section from the environment argument
\subsection{\homeworkSectionName} % Make a subsection with the custom name of the subsection
\enterProblemHeader{\homeworkProblemName\ [\homeworkSectionName]} % Header and footer within the environment
}{
\enterProblemHeader{\homeworkProblemName} % Header and footer after the environment
}

%----------------------------------------------------------------------------------------
%	NAME AND CLASS SECTION
%----------------------------------------------------------------------------------------

\newcommand{\hmwkTitle}{Redefine el hmwkTitle} % Assignment title
\newcommand{\hmwkDueDate}{asdfadsf} % Due date
\newcommand{\hmwkClass}{Redefine hmwkClass} % Course/class
\newcommand{\hmwkClassTime}{} % Class/lecture time
\newcommand{\hmwkClassInstructor}{} % Teacher/lecturer
\newcommand{\hmwkAuthorName}{Álvaro González Sotillo} % Your name

%----------------------------------------------------------------------------------------
%	TITLE PAGE
%----------------------------------------------------------------------------------------

\title{
\vspace{2in}
\textmd{\textbf{\hmwkClass:\ \hmwkTitle}}\\
\vspace{0.1in}\large{\textit{\hmwkClassInstructor\ \hmwkClassTime}}
\vspace{3in}
}

\author{\textbf{\hmwkAuthorName}}
\date{} % Insert date here if you want it to appear below your name


%----------------------------------------------------------------------------------------

\usepackage{fancybox}
\newcommand{\codigo}[1]{\texttt{#1}}


% CUADRITO
\newsavebox{\fmboxx}
\newenvironment{cuadrito}[1][14cm]
{\noindent \begin{center} \begin{lrbox}{\fmboxx}\noindent\begin{minipage}{#1}}
{\end{minipage}\end{lrbox}\noindent\shadowbox{\usebox{\fmboxx}} \end{center}}





\newenvironment{entradasalida}[2][14cm]
{
  \newcommand{\elnombredelafiguradeentradasalida}{#2}
  \begin{figure}[h]
    \begin{cuadrito}[#1]
      \begin{scriptsize}
\begin{alltt}
}
{%
\end{alltt}%
      \end{scriptsize}%
    \end{cuadrito}%
    \caption{\elnombredelafiguradeentradasalida}
  \end{figure}
}


\newenvironment{entradasalidacols}[2][14cm]
{
\newcommand{\elnombredelafiguradeentradasalida}{#2}
%\begin{wrapfigure}{}{0.1}
\begin{cuadrito}[#1]
\begin{scriptsize}
\begin{alltt}
}
{
\end{alltt}
\end{scriptsize}
\end{cuadrito}
\captionof{figure}{\elnombredelafiguradeentradasalida}
%\end{wrapfigure}
}


\usepackage{pgffor}
\newcommand{\ficheroautoref}[0]{%
  \foreach \ficherotex in {../../../common/plantilla-ejercicio.tex,../../common/plantilla-ejercicio.tex,../common/plantilla-ejercicio.tex,../apuntes/common/plantilla-ejercicio.tex}{

    \IfFileExists{\ficherotex}%
    {%
      \textattachfile[mimetype=text/plain,%
      description={La plantilla},%
      subject={La plantilla}]%
      {\ficherotex}%
      {}%
      % RECUPERO ESPACIO VERTICAL PERDIDO
      \vspace{-6em}%
    }%
    {}%
  }
  

  \textattachfile[mimetype=text/plain,
  description={El fichero TEX original para crear este documento, no sea que se nos pierda},
  subject={El fichero TEX original para crear este documento, no sea que se nos pierda}]
  {\currfilename}{}

}

\newcommand{\entradausuario}[1]{\textit{\textbf{#1}}}

\newcommand{\enlace}[2]{\textcolor{blue}{{\href{#1}{#2}}}}

\newcommand{\adjuntarfichero}[3]{
\textattachfile[mimetype=text/plain,
color={0 0 0},
description={#3},
subject={#1}]
{#1}
{\textcolor{blue}{\codigo{#2}}}
}

\newcommand{\adjuntardoc}[2]{%
\textattachfile[mimetype=text/plain,%
color={0 0 0},%
description={#1},%
subject={#1}]%
{#1}%
{\textcolor{blue}{#2}}%
}%


\newcommand{\plantilladeclase}[2]{
\adjuntarfichero{#1.java}{#2}{Plantilla para la clase #2}
}

% https://en.wikibooks.org/wiki/LaTeX/Source_Code_Listings#Encoding_issue
\lstset{literate=
  {á}{{\'a}}1 {é}{{\'e}}1 {í}{{\'i}}1 {ó}{{\'o}}1 {ú}{{\'u}}1
  {Á}{{\'A}}1 {É}{{\'E}}1 {Í}{{\'I}}1 {Ó}{{\'O}}1 {Ú}{{\'U}}1
  {à}{{\`a}}1 {è}{{\`e}}1 {ì}{{\`i}}1 {ò}{{\`o}}1 {ù}{{\`u}}1
  {À}{{\`A}}1 {È}{{\'E}}1 {Ì}{{\`I}}1 {Ò}{{\`O}}1 {Ù}{{\`U}}1
  {ä}{{\"a}}1 {ë}{{\"e}}1 {ï}{{\"i}}1 {ö}{{\"o}}1 {ü}{{\"u}}1
  {Ä}{{\"A}}1 {Ë}{{\"E}}1 {Ï}{{\"I}}1 {Ö}{{\"O}}1 {Ü}{{\"U}}1
  {â}{{\^a}}1 {ê}{{\^e}}1 {î}{{\^i}}1 {ô}{{\^o}}1 {û}{{\^u}}1
  {Â}{{\^A}}1 {Ê}{{\^E}}1 {Î}{{\^I}}1 {Ô}{{\^O}}1 {Û}{{\^U}}1
  {œ}{{\oe}}1 {Œ}{{\OE}}1 {æ}{{\ae}}1 {Æ}{{\AE}}1 {ß}{{\ss}}1
  {ű}{{\H{u}}}1 {Ű}{{\H{U}}}1 {ő}{{\H{o}}}1 {Ő}{{\H{O}}}1
  {ç}{{\c c}}1 {Ç}{{\c C}}1 {ø}{{\o}}1 {å}{{\r a}}1 {Å}{{\r A}}1
  {€}{{\euro}}1 {£}{{\pounds}}1 {«}{{\guillemotleft}}1
  {»}{{\guillemotright}}1 {ñ}{{\~n}}1 {Ñ}{{\~N}}1 {¿}{{?`}}1
}

\lstset{
  inputencoding=utf8,
  captionpos=b,
  frame=single,
  basicstyle=\small\ttfamily,
  showstringspaces=false,
  numbers=none,
  numbers=left,
  xleftmargin=2em,
  breaklines=true,
  postbreak=\mbox{\textcolor{red}{$\hookrightarrow$}\space}
}

\renewcommand{\lstlistingname}{Listado}
\captionsetup[lstlisting]{font={footnotesize},labelfont=bf,position=bottom}
\captionsetup[figure]{font={footnotesize},labelfont=bf}
\lstnewenvironment{listadojava}[2]
{
  \lstset{language=Java,caption={#1},label={#2}}
  \noindent\minipage{\linewidth}%
}
{\endminipage}

% LISTADO SHELL
\lstnewenvironment{listadoshell}[2]
{
  \lstset{language=bash,caption={#1},label={#2}}
  \noindent\minipage{\linewidth}%
}
{\endminipage}

% LISTADO TXT
\lstnewenvironment{listadotxt}[2]
{
  \lstset{caption={#1},label={#2}, keywords={}}
  \noindent\minipage{\linewidth}%
}
{\endminipage}

% LISTADO SQL
\lstnewenvironment{listadosql}[2]%
{%
  %(el 1 )caption es #1\\%
  %(el 2) label es #2\\%
  \lstset{language=sql,caption={#1},label={#2}}%
  \noindent\minipage{\linewidth}%
}
{\endminipage}
  






\newcommand{\graficosvg}[3][14cm]{
\begin{figure}[htbp]
\centering
\textattachfile{#2.svg}{
\color{black}
\includesvg[width=#1]{#2}
}
\caption{#3}
\end{figure}
}


\newcommand{\graficosvguml}[3][7cm]{
  \texttt{\graficosvg[#1]{#2}{#3}}
}


\newcommand{\primerapagina}{
\newpage
\ficheroautoref
\tableofcontents
\newpage
}

% CAJAS
\usepackage[skins]{tcolorbox}
\newtcolorbox{Aviso}[1][Aviso]{
  enhanced,
  colback=gray!5!white,
  colframe=gray!75!black,fonttitle=\bfseries,
  colbacktitle=gray!85!black,
  attach boxed title to top left={yshift=-2mm,xshift=2mm},
  title=#1
}


\newcommand{\StudentData}{
  \begin{cuadrito}[1\textwidth]
    \vspace{0.3cm}
    \large{
      \textbf{Apellidos:} \hrulefill \\
      \textbf{Nombre:} \hrulefill \\
      \textbf{Fecha:} \hrulefill \hspace{1cm} \textbf{Grupo:} \hrulefill \\
    }
    %\vspace{0.2cm}
  \end{cuadrito}
}

% TEXTO EN MONOESPACIO PERO QUE SE PARTE EN LINEAS
\usepackage{seqsplit}
\newcommand{\tecnico}[1]{\texttt{\seqsplit{#1}}}

% REEMPLAZAR TEXTO, NO LO USO
\def\replace#1#2#3{%
 \def\tmp##1#2{##1#3\tmp}%
   \tmp#1\stopreplace#2\stopreplace}
\def\stopreplace#1\stopreplace{}

\usepackage{eurosym}
\usepackage{needspace}




\renewcommand{\hmwkTitle}{Práctica PLSQL}
\renewcommand{\hmwkClass}{Bases de datos}

\usepackage{blindtext}

\begin{document}

% \maketitle

% ----------------------------------------------------------------------------------------
%	TABLE OF CONTENTS
% ----------------------------------------------------------------------------------------

% \setcounter{tocdepth}{1} % Uncomment this line if you don't want subsections listed in the ToC

\primerapagina

\setlength{\parindent}{1em}
\setlength{\parskip}{1em}

\section{Objetivo de la práctica}
En esta práctica el alumno utilizará las funcionalidades de PLSQL para automatizar algunas operaciones y para realizar comprobaciones sobre los datos. Estas operaciones y comprobaciones no pueden realizarse solo con SQL.


\newcommand{\maximopiezasavion}{15}
\newcommand{\maximopiezaspieza}{5}


\section{Descripción del problema}
Una compañía aeronáutica necesita informatizar la gestión de las piezas de sus aviones.
\begin{itemize}
\item Unas piezas se pueden formar de otras piezas, dando lugar a una pieza compuesta.
\item Se considera a un avión como una pieza compuesta.
%\item Cada avión está identificado por un número consecutivo.
\item Las piezas simples no están compuestas de otras piezas. Cada pieza simple tiene un \textit{serial number} único, y un \textit{part number}, que identifica el tipo de pieza.
\item Un avión puede tener tipos de piezas (\textit{part number}) repetidos (por ejemplo, 2 motores del mismo tipo).
\item Un avión no se compone nunca de más de {\maximopiezasavion} piezas simples en total.
\item Una pieza compuesta no se compone nunca de más de {\maximopiezaspieza} piezas simples \textit{directas}.
\end{itemize}

%%%%%%%%%%%%%%%%%%%%%%%%%%%%%%%%%%%%%%%%%%%%%%%%%%%%%%%%%%%%%%%%%%%
\begin{homeworkProblem}[: Definir el modelo de datos]
  Crea las tablas y funciones necesarias para soportar la descripción del problema. Las tablas y atributos concretos no son importantes, ya que el profesor consultará los datos a partir de la vista definida más adelante.
  
  \begin{itemize}
  \item Los aviones (y el resto de piezas) se darán de alta usando como identificador el valor de la secuencia \texttt{SECUENCIA\_PIEZA\_ID}, si la orden \texttt{INSERT} no especifica ningún valor.
    
  \item Las piezas simples tendrán un \textit{serial number} alfanumérico, que comienza por ``\texttt{SN-}'' y acaba con un número que se extrae de la secuencia \texttt{SECUENCIA\_SERIALNUMBER}
  \end{itemize}

\end{homeworkProblem}

%%%%%%%%%%%%%%%%%%%%%%%%%%%%%%%%%%%%%%%%%%%%%%%%%%%%%%%%%%%%%%%%%%%
\newcounter{Avion} \setcounter{Avion}{1}
\newcounter{SN} \setcounter{SN}{1}
\newcounter{Asiento} \setcounter{Asiento}{1}
\newcounter{Motor} \setcounter{Motor}{1}
\newcommand{\Contador}[1]{%
  \arabic{#1}%
  \stepcounter{#1}%
}
\newcommand{\FuselajePN}{PN-001}
\newcommand{\AsientoPN}{PN-002}
\newcommand{\ControlesPN}{PN-003}
\newcommand{\MotorPN}{PN-004}

\needspace{.15\textheight}
\begin{homeworkProblem}[: Procedimientos para crear y mostrar un avión (2 puntos)]
  Crea un procedimiento de nombre \texttt{CREAR\_AVION\_TIPO(avion\_id OUT number)} (listado \ref{lst:crear-avion-tipo}) que dé de alta un avión como el de la figura \ref{tabla:avion-tipo} y devuelva el \texttt{ID} del avión recién creado. Los \textit{serial number}, así como los identificadores de cada pieza, se extraerán de la secuencia correspondiente, y por eso no se muestran en los datos a insertar.

  
  \begin{figure}
    \begin{center}
      \begin{tabular}{|l|c|c|}
        \hline
        \textbf{Nombre de pieza} & \textbf{Part Number} \\
        \hline

        Avión \Contador{Avion} & \\

        \hspace{2em}Fuselaje                         & \FuselajePN \\
        \hspace{2em}Cabina                          & \\
        \hspace{4em}Asiento piloto                & \\
        \hspace{6em}Asiento \Contador{Asiento}   & \AsientoPN \\
        \hspace{6em}Controles                    & \ControlesPN \\
        \hspace{4em}Asiento \Contador{Asiento}     & \AsientoPN \\
        \hspace{4em}Asiento \Contador{Asiento}     & \AsientoPN \\
        \hspace{2em}Motor \Contador{Motor}           & \MotorPN \\
        \hspace{2em}Motor \Contador{Motor}           & \MotorPN \\
        \hline
      \end{tabular}
    \end{center}
    
    \caption{Datos del avión tipo}
    \label{tabla:avion-tipo}
  \end{figure}

  \begin{listadosql}{Creación del procedimiento \texttt{CREAR\_AVION\_TIPO}}{lst:crear-avion-tipo}
create or replace procedure CREAR_AVION_TIPO(avion_id OUT number)
as 
begin
  -- CONSIGUE EL NUEVO ID DEL AVION A PARTIR DE LA SECUENCIA
  -- INSERTA EL AVION CON ESE ID
  -- INSERTA LAS DEMÁS PIEZAS QUE TENGAN COMO PADRE AL AVION
  --   Y LAS PIEZAS QUE CUELGUEN DE ELLAS
end;
\end{listadosql}


  Crea un procedimiento \texttt{MUESTRA\_AVION(avion\_id number)} (listado \ref{lst:muestra-avion}), que imprima un avión y las piezas que lo componen, en forma de árbol. Usa \texttt{dbms\_output.put\_line()} para conseguir una salida parecida a la del listado \ref{lst:ejemplo-muestra-avion}.

\begin{listadosql}{Creación del procedimiento \texttt{MUESTRA\_AVION}}{lst:muestra-avion}
create or replace procedure MUESTRA_AVION( avion_id IN number ) 
as
begin
  -- MUESTRA EL ID Y EL NOMBRE DEL AVION
  -- POR CADA PIEZA QUE TENGA COMO PADRE AL ID DEL AVION:
  --   LLAMA A UN PRODEDIMIENTO QUE MUESTRE EL ID, ID DEL PADRE, NOMBRE, SN Y PN 
  --      Y MUESTRA A SUS PIEZAS HIJAS
end;
\end{listadosql}


\begin{listadotxt}{Ejemplo de salida de MUESTRA\_AVION}{lst:ejemplo-muestra-avion}
117 -  - Avion 1  
    118 - 117 - Fuselaje SN-89 PN-001
    119 - 117 - Cabina  
        120 - 119 - Asiento piloto  
            121 - 120 - Asiento 1 SN-90 PN-002
            122 - 120 - Controles SN-91 PN-003
        123 - 119 - Asiento 2 SN-92 PN-002
        124 - 119 - Asiento 3 SN-93 PN-002
    125 - 117 - Motor 1 SN-94 PN-004
    126 - 117 - Motor 2 SN-95 PN-004
    127 - 117 - Cabina  
        128 - 127 - Asiento piloto  
            129 - 128 - Asiento 1 SN-96 PN-002
            130 - 128 - Controles SN-97 PN-003
        131 - 127 - Asiento 2 SN-98 PN-002
        132 - 127 - Asiento 3 SN-99 PN-002

\end{listadotxt}
  
\end{homeworkProblem}

%%%%%%%%%%%%%%%%%%%%%%%%%%%%%%%%%%%%%%%%%%%%%%%%%%%%%%%%%%%%%%%%%%%
\begin{homeworkProblem}[: Vista para el profesor (1 punto)]

  Crea una vista \texttt{AVIONES\_PROFESOR} (listado \ref{lst:aviones-profesor}) que liste todas las piezas, aviones incluidos. Los tipos de las columnas serán los siguientes:
  
  \begin{center}
    \begin{tabular}{|c|c|p{10cm}|}
      \hline
      \textbf{Nombre columna} & \textbf{Tipo} & \textbf{Descripción} \\ 
      \hline
      \texttt{id} & \texttt{numeric} & Identificador de la pieza \\
      \texttt{padre\_id} & \texttt{numeric} & Identificador de la pieza padre, o \texttt{null} si es un avión\\
      \texttt{nombre} & \texttt{varchar} & Nombre de la pieza \\
      \texttt{sn} & \texttt{varchar} & \texttt{Serial number} de la pieza (\texttt{null} para piezas compuestas)\\
      \texttt{pn} & \texttt{varchar} & \texttt{Part number}, o tipo de pieza (\texttt{null} para piezas compuestas)\\
      \hline
    \end{tabular}
  \end{center}

  \begin{listadosql}{Creación de la vista \texttt{AVIONES\_PROFESOR}}{lst:aviones-profesor}
create or replace view AVIONES_PROFESOR(id, padre_id, nombre, sn, pn) as
select ...
\end{listadosql}

\end{homeworkProblem}

%%%%%%%%%%%%%%%%%%%%%%%%%%%%%%%%%%%%%%%%%%%%%%%%%%%%%%%%%%%%%%%%%%%
\needspace{.25\textheight}
\begin{homeworkProblem}[: Procedimientos de limpieza (1 punto)]
  \begin{itemize}
  \item Crea un procedimiento \texttt{BORRAR\_AVION} que reciba un identificador de avión y borre
    todas sus piezas. El procedimiento fallará si el avión no existe.
  \item Crea un procedimiento \texttt{BORRAR\_AVIONES} que borre todos los aviones (y posibles piezas
    sueltas) de la base de datos.
  \end{itemize}

  \begin{listadosql}{Creación de los procedimientos de limplieza}{lst:limpieza}
create or replace procedure BORRAR_AVION(avion_id IN number) as ... 
create or replace procedure BORRAR_AVIONES() as ...
\end{listadosql}

\end{homeworkProblem}

%%%%%%%%%%%%%%%%%%%%%%%%%%%%%%%%%%%%%%%%%%%%%%%%%%%%%%%%%%%%%%%%%%%
\needspace{.15\textheight}
\begin{homeworkProblem}[: Procedimientos de copia de piezas y aviones (3 puntos)]
  \begin{itemize}
  \item Crea un procedimiento \texttt{COPIAR\_PIEZA}
    que reciba un identificador de pieza y cree una copia de la misma (\textbf{con todas sus subpiezas}). La nueva pieza
    formará parte de la pieza \texttt{nueva\_pieza\_destino}. El parámetro \texttt{nueva\_pieza\_id} devolverá el
    identificador de la nueva pieza.

   
  \item Crea un procedimiento \texttt{COPIAR\_AVION} que reciba un identificador de avión y cree un avión con el mismo número de piezas y del mismo tipo que
    el original. El parámetro \texttt{copia\_avion\_id} devolverá el identificador del nuevo avión
  \end{itemize}

\begin{listadosql}{Cabecera y pseudocódigo de los procedimientos de copia}{lst:copia}
create or replace procedure COPIAR_PIEZA(
  pieza_id IN number, 
  nueva_pieza_destino IN number, 
  nueva_pieza_id OUT number) 
as
begin
  -- HAGO UNA COPIA DE LA PIEZA pieza_id, PERO CON nueva_pieza_destino COMO PADRE.
  -- PARA CADA PIEZA hija QUE TIENE COMO PADRE A pieza_id:
  --   HAGO UNA COPIA DE LA PIEZA hija QUE TENGA COMO PADRE A nueva_pieza_id,
  --   USANDO EL PROCEDIMIENTO COPIAR_PIEZA.
end; 

create or replace procedure COPIAR_AVION(
  avion_id IN number, 
  nuevo_avion_id OUT number) 
as
begin
  -- HAGO UNA COPIA DEL AVION Y DE SUS SUBPIEZAS.
  -- SEGURAMENTE UTILICE EL PROCEDIMIENTO COPIAR_PIEZA.
end;
\end{listadosql}

\end{homeworkProblem}

%%%%%%%%%%%%%%%%%%%%%%%%%%%%%%%%%%%%%%%%%%%%%%%%%%%%%%%%%%%%%%%%%%%
\needspace{.15\textheight}
\begin{homeworkProblem}[: Funciones de comprobación (Opcional, 3 puntos)]
  \begin{itemize}

  \item La función \texttt{COMPRUEBA\_PIEZA(ID)} provocará un error en los siguientes casos:
    \begin{itemize}
    \item si la pieza tiene más de {\maximopiezaspieza} piezas simples
    \item el \texttt{ID} es el de un avión.
    \item el \texttt{ID} no existe.
    \end{itemize}

  \item La función \texttt{COMPRUEBA\_AVION(ID)} provocará un error en los siguientes casos:
    \begin{itemize}
    \item el avión tiene más de {\maximopiezasavion} piezas simples
    \item el avión tiene alguna pieza compuesta con más de {\maximopiezaspieza} piezas simples.  
    \item si el \texttt{ID} no es el de un avión.
    \item si el \texttt{ID} no existe.  
    \end{itemize}
  \end{itemize}

  
\begin{listadosql}{Creación y pseudocódigo de funciones de comprobación}{label}

create or replace function COMPRUEBA_PIEZA(id IN number) return varchar 
as
begin
  -- SI LA PIEZA TIENE MUCHAS PIEZAS SIMPLES, O ES UN AVION, O NO EXISTE,
  --   HAGO UN RAISE_APPLICATION_ERROR
  -- SI NO, DEVUELVO 'ok'
end; 

create or replace function COMPRUEBA_AVION(id IN number) return varchar 
as
begin
  -- SI EL AVION TIENE MUCHAS PIEZAS SIMPLES, O NO ES UN AVION, O NO EXISTE, 
  --   HAGO UN RAISE_APPLICATION_ERROR
  -- COMPRUEBO LAS PIEZAS DEL AVION CON COMPRUEBA_PIEZA
  -- SI NO, DEVUELVO 'ok'
end; 

\end{listadosql}

    Nota: puede ser muy útil una función que reciba un identificador de pieza y devuelva el número de piezas simples de las que se compone. La sentencia propietaria de Oracle \texttt{CONNECT BY PRIOR} puede facilitar mucho esta función (aunque no es necesaria).


\end{homeworkProblem}

\section{Instrucciones de entrega}
Se creará un único fichero SQL para todos los apartados. Este fichero se corregirá de forma semiautomática, por lo que es necesario seguir la nomenclatura propuesta en el ejercicio.

La autoría del trabajo puede ser por parejas. A pesar de ello, cada alumno debe subir al moodle una copia del trabajo. Los alumnos que no dispongan de usuario de educamadrid en el momento de la entrega pueden enviar su trabajo a \enlace{alvaro.gonzalezsotillo@educa.madrid.org}{alvaro.gonzalezsotillo@educa.madrid.org}


Sube el documento a \enlace{http://aulavirtual2.educa.madrid.org/mod/assignment/view.php?id=1116703}{la tarea correspondiente en el aula virtual}
Presta atención al plazo de entrega (con fecha y hora).


\end{document}




