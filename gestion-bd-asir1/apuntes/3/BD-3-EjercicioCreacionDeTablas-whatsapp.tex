%%%%%%%%%%%%%%%%%%%%%%%%%%%%%%%%%%%%%%%%%
% Programming/Coding Assignment
% LaTeX Template
%
% This template has been downloaded from:
% http://www.latextemplates.com
%
% Original author:
% Ted Pavlic (http://www.tedpavlic.com)
%
% Note:
% The \lipsum[#] commands throughout this template generate dummy text
% to fill the template out. These commands should all be removed when 
% writing assignment content.
%
% This template uses a Perl script as an example snippet of code, most other
% languages are also usable. Configure them in the "CODE INCLUSION 
% CONFIGURATION" section.
%
%%%%%%%%%%%%%%%%%%%%%%%%%%%%%%%%%%%%%%%%%

%----------------------------------------------------------------------------------------
%	PACKAGES AND OTHER DOCUMENT CONFIGURATIONS
%----------------------------------------------------------------------------------------

\documentclass[a4paper]{article}
\usepackage[utf8]{inputenc}
\usepackage{listingsutf8}
\usepackage[spanish]{babel}




\usepackage{fancyhdr} % Required for custom headers
\usepackage{lastpage} % Required to determine the last page for the footer
\usepackage{extramarks} % Required for headers and footers
\usepackage[usenames,dvipsnames]{color} % Required for custom colors
\usepackage{graphicx} % Required to insert images
\usepackage{listings} % Required for insertion of code
\usepackage{courier} % Required for the courier font
\usepackage{lipsum} % Used for inserting dummy 'Lorem ipsum' text into the template
\usepackage{svg}
\usepackage{attachfile}
\usepackage{currfile}
\usepackage{multicol}
\usepackage{alltt}
\usepackage{framed}
\usepackage{caption}
\usepackage{seqsplit}

\hypersetup{
colorlinks,
citecolor=black,
filecolor=black,
linkcolor=black,
urlcolor=blue
}


% Margins
\topmargin=-0.45in
\evensidemargin=0in
\oddsidemargin=0in
\textwidth=6.5in
\textheight=9.8in
\headsep=0.25in

\linespread{1.1} % Line spacing
\setlength{\parindent}{1.5em}
\setlength{\parskip}{1em}


% Set up the header and footer
\pagestyle{fancy}
%\lhead{\hmwkAuthorName} % Top left header
\lhead{\hmwkClass}
\rhead{\hmwkTitle} % Top right header
\chead{} % Top center head
\lfoot{\lastxmark} % Bottom left footer
\cfoot{} % Bottom center footer
\rfoot{ \thepage\ / \protect\pageref{LastPage}} % Bottom right footer
\renewcommand\headrulewidth{0.4pt} % Size of the header rule
\renewcommand\footrulewidth{0.4pt} % Size of the footer rule



%----------------------------------------------------------------------------------------
%	CODE INCLUSION CONFIGURATION
%----------------------------------------------------------------------------------------
\renewcommand{\ttdefault}{pcr}
\definecolor{MyDarkGreen}{rgb}{0.0,0.4,0.0} % This is the color used for comments
\lstloadlanguages{Perl} % Load Perl syntax for listings, for a list of other languages supported see: ftp://ftp.tex.ac.uk/tex-archive/macros/latex/contrib/listings/listings.pdf
\lstset{
language=sh, % Use Perl in this example
frame=single, % Single frame around code
basicstyle=\small\ttfamily, % Use small true type font
keywordstyle=[1]\small\color{Blue}\bf, % Perl functions bold and blue
keywordstyle=[2]\small\color{Purple}, % Perl function arguments purple
keywordstyle=[3]\small\color{Blue}\underbar, % Custom functions underlined and blue
identifierstyle=, % Nothing special about identifiers                                         
commentstyle=\small\color{MyDarkGreen}, % Comments small dark green courier font
stringstyle=\small\color{Purple}, % Strings are purple
showstringspaces=false, % Don't put marks in string spaces
tabsize=5, % 5 spaces per tab
morecomment=[l][\small\color{Blue}]{...}, % Line continuation (...) like blue comment
numbers=none, % Line numbers on left
firstnumber=1, % Line numbers start with line 1
numberstyle=\tiny\color{Blue}, % Line numbers are blue and small
stepnumber=0 % Line numbers go in steps of 5
}

% Creates a new command to include a perl script, the first parameter is the filename of the script (without .pl), the second parameter is the caption
\newcommand{\perlscript}[2]{
\begin{itemize}
\item[]\lstinputlisting[caption=#2,label=#1]{#1.pl}
\end{itemize}
}

%----------------------------------------------------------------------------------------
%	DOCUMENT STRUCTURE COMMANDS
%	Skip this unless you know what you're doing
%----------------------------------------------------------------------------------------

% Header and footer for when a page split occurs within a problem environment
\newcommand{\enterProblemHeader}[1]{
%\nobreak\extramarks{#1}{#1 continued on next page\ldots}\nobreak
%\nobreak\extramarks{#1 (continued)}{#1 continued on next page\ldots}\nobreak
}

% Header and footer for when a page split occurs between problem environments
\newcommand{\exitProblemHeader}[1]{
%\nobreak\extramarks{#1 (continued)}{#1 continued on next page\ldots}\nobreak
%\nobreak\extramarks{#1}{}\nobreak
}

\setcounter{secnumdepth}{0} % Removes default section numbers
\newcounter{homeworkProblemCounter} % Creates a counter to keep track of the number of problems

\newcommand{\homeworkProblemName}{}
\newenvironment{homeworkProblem}[1][]{ % Makes a new environment called homeworkProblem which takes 1 argument (custom name) but the default is "Problem #"
\stepcounter{homeworkProblemCounter} % Increase counter for number of problems
\renewcommand{\homeworkProblemName}{Ejercicio \arabic{homeworkProblemCounter} #1} % Assign \homeworkProblemName the name of the problem
\section{\homeworkProblemName} % Make a section in the document with the custom problem count
\enterProblemHeader{\homeworkProblemName} % Header and footer within the environment
}{
\exitProblemHeader{\homeworkProblemName} % Header and footer after the environment
%\clearpage
}


\newcommand{\problemAnswer}[1]{ % Defines the problem answer command with the content as the only argument
\noindent\framebox[\columnwidth][c]{\begin{minipage}{0.98\columnwidth}#1\end{minipage}} % Makes the box around the problem answer and puts the content inside
}

\newcommand{\homeworkSectionName}{}
\newenvironment{homeworkSection}[1]{ % New environment for sections within homework problems, takes 1 argument - the name of the section
\renewcommand{\homeworkSectionName}{#1} % Assign \homeworkSectionName to the name of the section from the environment argument
\subsection{\homeworkSectionName} % Make a subsection with the custom name of the subsection
\enterProblemHeader{\homeworkProblemName\ [\homeworkSectionName]} % Header and footer within the environment
}{
\enterProblemHeader{\homeworkProblemName} % Header and footer after the environment
}

%----------------------------------------------------------------------------------------
%	NAME AND CLASS SECTION
%----------------------------------------------------------------------------------------

\newcommand{\hmwkTitle}{Redefine el hmwkTitle} % Assignment title
\newcommand{\hmwkDueDate}{asdfadsf} % Due date
\newcommand{\hmwkClass}{Redefine hmwkClass} % Course/class
\newcommand{\hmwkClassTime}{} % Class/lecture time
\newcommand{\hmwkClassInstructor}{} % Teacher/lecturer
\newcommand{\hmwkAuthorName}{Álvaro González Sotillo} % Your name

%----------------------------------------------------------------------------------------
%	TITLE PAGE
%----------------------------------------------------------------------------------------

\title{
\vspace{2in}
\textmd{\textbf{\hmwkClass:\ \hmwkTitle}}\\
\vspace{0.1in}\large{\textit{\hmwkClassInstructor\ \hmwkClassTime}}
\vspace{3in}
}

\author{\textbf{\hmwkAuthorName}}
\date{} % Insert date here if you want it to appear below your name


%----------------------------------------------------------------------------------------

\usepackage{fancybox}
\newcommand{\codigo}[1]{\texttt{#1}}


% CUADRITO
\newsavebox{\fmboxx}
\newenvironment{cuadrito}[1][14cm]
{\noindent \begin{center} \begin{lrbox}{\fmboxx}\noindent\begin{minipage}{#1}}
{\end{minipage}\end{lrbox}\noindent\shadowbox{\usebox{\fmboxx}} \end{center}}





\newenvironment{entradasalida}[2][14cm]
{
  \newcommand{\elnombredelafiguradeentradasalida}{#2}
  \begin{figure}[h]
    \begin{cuadrito}[#1]
      \begin{scriptsize}
\begin{alltt}
}
{%
\end{alltt}%
      \end{scriptsize}%
    \end{cuadrito}%
    \caption{\elnombredelafiguradeentradasalida}
  \end{figure}
}


\newenvironment{entradasalidacols}[2][14cm]
{
\newcommand{\elnombredelafiguradeentradasalida}{#2}
%\begin{wrapfigure}{}{0.1}
\begin{cuadrito}[#1]
\begin{scriptsize}
\begin{alltt}
}
{
\end{alltt}
\end{scriptsize}
\end{cuadrito}
\captionof{figure}{\elnombredelafiguradeentradasalida}
%\end{wrapfigure}
}


\usepackage{pgffor}
\newcommand{\ficheroautoref}[0]{%
  \foreach \ficherotex in {../../../common/plantilla-ejercicio.tex,../../common/plantilla-ejercicio.tex,../common/plantilla-ejercicio.tex,../apuntes/common/plantilla-ejercicio.tex}{

    \IfFileExists{\ficherotex}%
    {%
      \textattachfile[mimetype=text/plain,%
      description={La plantilla},%
      subject={La plantilla}]%
      {\ficherotex}%
      {}%
      % RECUPERO ESPACIO VERTICAL PERDIDO
      \vspace{-6em}%
    }%
    {}%
  }
  

  \textattachfile[mimetype=text/plain,
  description={El fichero TEX original para crear este documento, no sea que se nos pierda},
  subject={El fichero TEX original para crear este documento, no sea que se nos pierda}]
  {\currfilename}{}

}

\newcommand{\entradausuario}[1]{\textit{\textbf{#1}}}

\newcommand{\enlace}[2]{\textcolor{blue}{{\href{#1}{#2}}}}

\newcommand{\adjuntarfichero}[3]{
\textattachfile[mimetype=text/plain,
color={0 0 0},
description={#3},
subject={#1}]
{#1}
{\textcolor{blue}{\codigo{#2}}}
}

\newcommand{\adjuntardoc}[2]{%
\textattachfile[mimetype=text/plain,%
color={0 0 0},%
description={#1},%
subject={#1}]%
{#1}%
{\textcolor{blue}{#2}}%
}%


\newcommand{\plantilladeclase}[2]{
\adjuntarfichero{#1.java}{#2}{Plantilla para la clase #2}
}

% https://en.wikibooks.org/wiki/LaTeX/Source_Code_Listings#Encoding_issue
\lstset{literate=
  {á}{{\'a}}1 {é}{{\'e}}1 {í}{{\'i}}1 {ó}{{\'o}}1 {ú}{{\'u}}1
  {Á}{{\'A}}1 {É}{{\'E}}1 {Í}{{\'I}}1 {Ó}{{\'O}}1 {Ú}{{\'U}}1
  {à}{{\`a}}1 {è}{{\`e}}1 {ì}{{\`i}}1 {ò}{{\`o}}1 {ù}{{\`u}}1
  {À}{{\`A}}1 {È}{{\'E}}1 {Ì}{{\`I}}1 {Ò}{{\`O}}1 {Ù}{{\`U}}1
  {ä}{{\"a}}1 {ë}{{\"e}}1 {ï}{{\"i}}1 {ö}{{\"o}}1 {ü}{{\"u}}1
  {Ä}{{\"A}}1 {Ë}{{\"E}}1 {Ï}{{\"I}}1 {Ö}{{\"O}}1 {Ü}{{\"U}}1
  {â}{{\^a}}1 {ê}{{\^e}}1 {î}{{\^i}}1 {ô}{{\^o}}1 {û}{{\^u}}1
  {Â}{{\^A}}1 {Ê}{{\^E}}1 {Î}{{\^I}}1 {Ô}{{\^O}}1 {Û}{{\^U}}1
  {œ}{{\oe}}1 {Œ}{{\OE}}1 {æ}{{\ae}}1 {Æ}{{\AE}}1 {ß}{{\ss}}1
  {ű}{{\H{u}}}1 {Ű}{{\H{U}}}1 {ő}{{\H{o}}}1 {Ő}{{\H{O}}}1
  {ç}{{\c c}}1 {Ç}{{\c C}}1 {ø}{{\o}}1 {å}{{\r a}}1 {Å}{{\r A}}1
  {€}{{\euro}}1 {£}{{\pounds}}1 {«}{{\guillemotleft}}1
  {»}{{\guillemotright}}1 {ñ}{{\~n}}1 {Ñ}{{\~N}}1 {¿}{{?`}}1
}

\lstset{
  inputencoding=utf8,
  captionpos=b,
  frame=single,
  basicstyle=\small\ttfamily,
  showstringspaces=false,
  numbers=none,
  numbers=left,
  xleftmargin=2em,
  breaklines=true,
  postbreak=\mbox{\textcolor{red}{$\hookrightarrow$}\space}
}

\renewcommand{\lstlistingname}{Listado}
\captionsetup[lstlisting]{font={footnotesize},labelfont=bf,position=bottom}
\captionsetup[figure]{font={footnotesize},labelfont=bf}
\lstnewenvironment{listadojava}[2]
{
  \lstset{language=Java,caption={#1},label={#2}}
  \noindent\minipage{\linewidth}%
}
{\endminipage}

% LISTADO SHELL
\lstnewenvironment{listadoshell}[2]
{
  \lstset{language=bash,caption={#1},label={#2}}
  \noindent\minipage{\linewidth}%
}
{\endminipage}

% LISTADO TXT
\lstnewenvironment{listadotxt}[2]
{
  \lstset{caption={#1},label={#2}, keywords={}}
  \noindent\minipage{\linewidth}%
}
{\endminipage}

% LISTADO SQL
\lstnewenvironment{listadosql}[2]%
{%
  %(el 1 )caption es #1\\%
  %(el 2) label es #2\\%
  \lstset{language=sql,caption={#1},label={#2}}%
  \noindent\minipage{\linewidth}%
}
{\endminipage}
  






\newcommand{\graficosvg}[3][14cm]{
\begin{figure}[htbp]
\centering
\textattachfile{#2.svg}{
\color{black}
\includesvg[width=#1]{#2}
}
\caption{#3}
\end{figure}
}


\newcommand{\graficosvguml}[3][7cm]{
  \texttt{\graficosvg[#1]{#2}{#3}}
}


\newcommand{\primerapagina}{
\newpage
\ficheroautoref
\tableofcontents
\newpage
}

% CAJAS
\usepackage[skins]{tcolorbox}
\newtcolorbox{Aviso}[1][Aviso]{
  enhanced,
  colback=gray!5!white,
  colframe=gray!75!black,fonttitle=\bfseries,
  colbacktitle=gray!85!black,
  attach boxed title to top left={yshift=-2mm,xshift=2mm},
  title=#1
}


\newcommand{\StudentData}{
  \begin{cuadrito}[1\textwidth]
    \vspace{0.3cm}
    \large{
      \textbf{Apellidos:} \hrulefill \\
      \textbf{Nombre:} \hrulefill \\
      \textbf{Fecha:} \hrulefill \hspace{1cm} \textbf{Grupo:} \hrulefill \\
    }
    %\vspace{0.2cm}
  \end{cuadrito}
}

% TEXTO EN MONOESPACIO PERO QUE SE PARTE EN LINEAS
\usepackage{seqsplit}
\newcommand{\tecnico}[1]{\texttt{\seqsplit{#1}}}

% REEMPLAZAR TEXTO, NO LO USO
\def\replace#1#2#3{%
 \def\tmp##1#2{##1#3\tmp}%
   \tmp#1\stopreplace#2\stopreplace}
\def\stopreplace#1\stopreplace{}

\usepackage{eurosym}

\newcommand{\anio}{2019}




\renewcommand{\hmwkTitle}{Diseño E/R, paso a SQL e inserción de datos}
\renewcommand{\hmwkClass}{Gestión de Bases de Datos}

\usepackage{enumitem}% http://ctan.org/pkg/enumitem
%\setlist[itemize]{parsep=0em}

\begin{document}

% \maketitle

% ----------------------------------------------------------------------------------------
%	TABLE OF CONTENTS
% ----------------------------------------------------------------------------------------

% \setcounter{tocdepth}{1} % Uncomment this line if you don't want subsections listed in the ToC

\primerapagina


\section{Objetivo de la práctica}
Se pretende que el alumno sea capaz de pasar desde las necesidades de datos de una empresa hasta las órdenes SQL necesarias para implementar dicha necesidad en Oracle

Se puede descargar \enlace{https://alvarogonzalezsotillo.github.io/apuntes-clase/gestion-bd-asir1/apuntes/3/BD-3-EjercicioCreacionDeTablas-whatsapp.pdf}{la última versión de la práctica en este enlace}

\section{Modelo de datos}
Una compañía decide lanzar un sistema de mensajería al estilo de Whatsapp.
\begin{itemize}
\item De cada usuario se conoce su número de teléfono y un \textit{nick}.
\item Cada usuario mantiene una lista de contactos, que son otros usuarios.
\item Los usuarios pueden enviar mensajes de texto a cualquiera de sus contactos. Un mensaje es un texto de hasta 1000 caracteres.
\item Los mensajes tienen dos confirmaciones:
  \begin{itemize}
  \item Confirmación de enviado: el mensaje ha llegado al terminal del destinatario
  \item Confirmación de leido: el usuario ha abierto el mensaje
  \end{itemize}
  
  
\item Los usuarios pueden crear grupos de \textit{chat}:
  \begin{itemize}
  \item Un grupo tiene un nombre
  \item El usuario creador es el administrador, y agrega a los usuarios que desea.
  \item Un mensaje enviado a un grupo de \textit{chat} es recibido por todos sus integrantes (menos el que lo envía)
  \item Las confirmaciones de los mensajes de un \textit{chat} son para cada usuario del \textit{chat}
  \end{itemize}
  
\end{itemize}



\begin{homeworkProblem}[: Realizar el diagrama \textbf{ER} con Oracle \textbf{SQLDeveloper} (2 puntos)]
  En el diagrama deben ser visibles (al menos):
  \begin{itemize}
  \item Los nombres de las entidades
  \item Los atributos de las entidades
  \item Las relaciones entre las entidades
  \end{itemize}

\end{homeworkProblem}


\begin{homeworkProblem}[: Crear las tablas de Oracle utilizando \textbf{SQL} (1 punto)]
  Se entregará un fichero de texto de extensión \texttt{.SQL} con las órdenes \textbf{SQL} de creación de tablas, sus claves primarias, extranjeras, valores por defecto y restricciones.

\end{homeworkProblem}



\begin{homeworkProblem}[: Insertar datos en las tablas (2 puntos)]

  El cliente quiere insertar los siguientes datos para comprobar la idoneidad del modelo relacional:
  
  \begin{tabular}{|l|l|}
    \hline
    \multicolumn{2}{|c|}{Usuarios} \\
    \hline
    Teléfono & Nick \\
    \hline
    11111111 & Pepe \\
    22222222 & María \\
    33333333 & Juan \\
    44444444 & Susana \\
    \hline
  \end{tabular}
  \begin{tabular}{|l|l|}
    \hline
    \multicolumn{2}{|c|}{Contactos} \\
    \hline
    Nick & Sus contactos\\
    \hline
    María & Pepe \\
    Pepe & María, Juan, Susana \\
    Juan & Pepe, Susana \\
    Susana & Juan \\
    \hline
  \end{tabular}
  
  \begin{tabular}{|l|l|l|l|l|}
    \hline
    \multicolumn{5}{|c|}{Mensajes} \\
    \hline
    Remitente & Destinatario & Recibido & Leído & Mensaje \\
    \hline
    Pepe & María & Sí & Sí & Vamos a hacer un grupo de ASIR \\
    María & Pepe & Sí & Sí & Vale \\
    Susana & Juan & Sí & No & ¿Te apuntas a la nieve? \\
    Juan & Susana & No & No & Me apunto \\
    \hline
  \end{tabular}
  
  \begin{tabular}{|l|l|l|}
    \hline
    \multicolumn{3}{|c|}{\textit{Chats}} \\
    \hline
    Nombre de \textit{chat} & Administrador & Integrantes adicionales \\
    \hline
    Cosas de ASIR & Pepe & María, Juan, Susana \\
    Viaje a la nieve & Juan & Pepe, Susana \\
    \hline
  \end{tabular}

  
  \begin{tabular}{|l|l|l|l|l|}
    \hline
    \multicolumn{5}{|c|}{Mensajes en \textit{chats}} \\
    \hline
    Nombre de \textit{chat} & Remitente & Mensaje & Recibido & Leido \\
    \hline
    Cosas de Asir & Pepe & Hoy el profesor no viene & Recibido por todos & Leído por todos \\
    Cosas de Asir & María & Pues ya estoy en el insti :( & Recibido por todos & Nadie lo ha leído \\
    Viaje a la nieve & Juan & El sábado no puedo & Recibido solo por Pepe & Leido solo por Pepe \\
    \hline
  \end{tabular}
  
  Es importante que los datos coincidan con los especificados. Los valores \textit{cierto} y \textit{falso} se pondrán como una cadena con \texttt{si} y \texttt{no}.
  

  Para comprobar los nuevos datos introducidos, se consultarán las siguientes vistas:
  
  \begin{itemize}
  \item \texttt{V\_CONTACTOS(nombreusuario, nombrecontacto)}
  
  \item \texttt{V\_MENSAJESRECIBIDOSPOR(nombredestinatario, nombreemisor, textomensaje, recibido, leido)}: Todos los mensajes recibidos por un destinatario, sean directos o dentro de un grupo de \texttt{chat}
  \item \texttt{V\_MENSAJESPORGRUPO(nombrechat, nombreemisor, textomensaje, cuantosrecibido, cuantosleido)}: Mensajes enviados a un grupo de \textit{chat}
    
  \item \texttt{V\_GRUPOSDECHAT(nombrechat,integrantes)}: Número de integrantes de un chat, incluido el creador

  \begin{listadosql}{Vistas a crear}{lst:vistas}
  create view V_CONTACTOS(nombreusuario, nombrecontacto) as
  select ...
  
  create view V_MENSAJESRECIBIDOSPOR(nombredestinatario,nombreemisor,textomensaje,recibido,leido) as
  select ...
  
  create view V_MENSAJESPORGRUPO(nombrechat,nombreemisor,textomensaje,cuantosrecibido,cuantosleido) as
  select ...
    
  create view V_GRUPOSDECHAT(nombrechat,integrantes) as
  select ...
\end{listadosql}
    
\end{itemize}

\end{homeworkProblem}

\section{Instrucciones de entrega}
\begin{itemize}
\item El ejercicio se realizará y entregará de manera individual.
  \begin{itemize}
  \item Solo se admiten trabajos en pareja, si en clase es necesario compartir ordenador.
  \end{itemize}
\item Entrega tu trabajo en un fichero \texttt{ZIP}:
  \begin{itemize}
  \item La imagen de tu modelo ER  con SQLDeveloper: \texttt{\textbf{1.modelo-er.pdf}}
  \item El script de creación de tablas: \texttt{\textbf{2.creacion.sql}}
  \item El script de inserción de datos en las tablas: \texttt{\textbf{3.insercion.sql}}
  \item El script de creación de vistas: \texttt{\textbf{4.vistas.sql}}
  \end{itemize}
\item Los nombres de los ficheros incluyen un número (para que estén ordenados). No incluyas mayúsculas. No incluyas acentos. No pongas espacios. No añadas palabras, letras ni números a los nombres.
  \item La corrección se realizará de forma semiautomática. Es \textbf{importante} que los nombres de ficheros, vistas y atributos de vistas sean los especificados.
\item Sube el documento a \enlace{https://aulavirtual3.educa.madrid.org/ies.alonsodeavellan.alcala/mod/assign/view.php?id=12774}{la tarea correspondiente en el aula virtual}
\item Presta atención al plazo de entrega (con fecha y hora).
  
\end{itemize}
\end{document}




