%%%%%%%%%%%%%%%%%%%%%%%%%%%%%%%%%%%%%%%%%
% Programming/Coding Assignment
% LaTeX Template
%
% This template has been downloaded from:
% http://www.latextemplates.com
%
% Original author:
% Ted Pavlic (http://www.tedpavlic.com)
%
% Note:
% The \lipsum[#] commands throughout this template generate dummy text
% to fill the template out. These commands should all be removed when 
% writing assignment content.
%
% This template uses a Perl script as an example snippet of code, most other
% languages are also usable. Configure them in the "CODE INCLUSION 
% CONFIGURATION" section.
%
%%%%%%%%%%%%%%%%%%%%%%%%%%%%%%%%%%%%%%%%%

%----------------------------------------------------------------------------------------
%	PACKAGES AND OTHER DOCUMENT CONFIGURATIONS
%----------------------------------------------------------------------------------------

\documentclass[a4paper]{article}
\usepackage[utf8]{inputenc}
\usepackage{listingsutf8}
\usepackage[spanish]{babel}




\usepackage{fancyhdr} % Required for custom headers
\usepackage{lastpage} % Required to determine the last page for the footer
\usepackage{extramarks} % Required for headers and footers
\usepackage[usenames,dvipsnames]{color} % Required for custom colors
\usepackage{graphicx} % Required to insert images
\usepackage{listings} % Required for insertion of code
\usepackage{courier} % Required for the courier font
\usepackage{lipsum} % Used for inserting dummy 'Lorem ipsum' text into the template
\usepackage{svg}
\usepackage{attachfile}
\usepackage{currfile}
\usepackage{multicol}
\usepackage{alltt}
\usepackage{framed}
\usepackage{caption}
\usepackage{seqsplit}

\hypersetup{
colorlinks,
citecolor=black,
filecolor=black,
linkcolor=black,
urlcolor=blue
}


% Margins
\topmargin=-0.45in
\evensidemargin=0in
\oddsidemargin=0in
\textwidth=6.5in
\textheight=9.8in
\headsep=0.25in

\linespread{1.1} % Line spacing
\setlength{\parindent}{1.5em}
\setlength{\parskip}{1em}


% Set up the header and footer
\pagestyle{fancy}
%\lhead{\hmwkAuthorName} % Top left header
\lhead{\hmwkClass}
\rhead{\hmwkTitle} % Top right header
\chead{} % Top center head
\lfoot{\lastxmark} % Bottom left footer
\cfoot{} % Bottom center footer
\rfoot{ \thepage\ / \protect\pageref{LastPage}} % Bottom right footer
\renewcommand\headrulewidth{0.4pt} % Size of the header rule
\renewcommand\footrulewidth{0.4pt} % Size of the footer rule



%----------------------------------------------------------------------------------------
%	CODE INCLUSION CONFIGURATION
%----------------------------------------------------------------------------------------
\renewcommand{\ttdefault}{pcr}
\definecolor{MyDarkGreen}{rgb}{0.0,0.4,0.0} % This is the color used for comments
\lstloadlanguages{Perl} % Load Perl syntax for listings, for a list of other languages supported see: ftp://ftp.tex.ac.uk/tex-archive/macros/latex/contrib/listings/listings.pdf
\lstset{
language=sh, % Use Perl in this example
frame=single, % Single frame around code
basicstyle=\small\ttfamily, % Use small true type font
keywordstyle=[1]\small\color{Blue}\bf, % Perl functions bold and blue
keywordstyle=[2]\small\color{Purple}, % Perl function arguments purple
keywordstyle=[3]\small\color{Blue}\underbar, % Custom functions underlined and blue
identifierstyle=, % Nothing special about identifiers                                         
commentstyle=\small\color{MyDarkGreen}, % Comments small dark green courier font
stringstyle=\small\color{Purple}, % Strings are purple
showstringspaces=false, % Don't put marks in string spaces
tabsize=5, % 5 spaces per tab
morecomment=[l][\small\color{Blue}]{...}, % Line continuation (...) like blue comment
numbers=none, % Line numbers on left
firstnumber=1, % Line numbers start with line 1
numberstyle=\tiny\color{Blue}, % Line numbers are blue and small
stepnumber=0 % Line numbers go in steps of 5
}

% Creates a new command to include a perl script, the first parameter is the filename of the script (without .pl), the second parameter is the caption
\newcommand{\perlscript}[2]{
\begin{itemize}
\item[]\lstinputlisting[caption=#2,label=#1]{#1.pl}
\end{itemize}
}

%----------------------------------------------------------------------------------------
%	DOCUMENT STRUCTURE COMMANDS
%	Skip this unless you know what you're doing
%----------------------------------------------------------------------------------------

% Header and footer for when a page split occurs within a problem environment
\newcommand{\enterProblemHeader}[1]{
%\nobreak\extramarks{#1}{#1 continued on next page\ldots}\nobreak
%\nobreak\extramarks{#1 (continued)}{#1 continued on next page\ldots}\nobreak
}

% Header and footer for when a page split occurs between problem environments
\newcommand{\exitProblemHeader}[1]{
%\nobreak\extramarks{#1 (continued)}{#1 continued on next page\ldots}\nobreak
%\nobreak\extramarks{#1}{}\nobreak
}

\setcounter{secnumdepth}{0} % Removes default section numbers
\newcounter{homeworkProblemCounter} % Creates a counter to keep track of the number of problems

\newcommand{\homeworkProblemName}{}
\newenvironment{homeworkProblem}[1][]{ % Makes a new environment called homeworkProblem which takes 1 argument (custom name) but the default is "Problem #"
\stepcounter{homeworkProblemCounter} % Increase counter for number of problems
\renewcommand{\homeworkProblemName}{Ejercicio \arabic{homeworkProblemCounter} #1} % Assign \homeworkProblemName the name of the problem
\section{\homeworkProblemName} % Make a section in the document with the custom problem count
\enterProblemHeader{\homeworkProblemName} % Header and footer within the environment
}{
\exitProblemHeader{\homeworkProblemName} % Header and footer after the environment
%\clearpage
}


\newcommand{\problemAnswer}[1]{ % Defines the problem answer command with the content as the only argument
\noindent\framebox[\columnwidth][c]{\begin{minipage}{0.98\columnwidth}#1\end{minipage}} % Makes the box around the problem answer and puts the content inside
}

\newcommand{\homeworkSectionName}{}
\newenvironment{homeworkSection}[1]{ % New environment for sections within homework problems, takes 1 argument - the name of the section
\renewcommand{\homeworkSectionName}{#1} % Assign \homeworkSectionName to the name of the section from the environment argument
\subsection{\homeworkSectionName} % Make a subsection with the custom name of the subsection
\enterProblemHeader{\homeworkProblemName\ [\homeworkSectionName]} % Header and footer within the environment
}{
\enterProblemHeader{\homeworkProblemName} % Header and footer after the environment
}

%----------------------------------------------------------------------------------------
%	NAME AND CLASS SECTION
%----------------------------------------------------------------------------------------

\newcommand{\hmwkTitle}{Redefine el hmwkTitle} % Assignment title
\newcommand{\hmwkDueDate}{asdfadsf} % Due date
\newcommand{\hmwkClass}{Redefine hmwkClass} % Course/class
\newcommand{\hmwkClassTime}{} % Class/lecture time
\newcommand{\hmwkClassInstructor}{} % Teacher/lecturer
\newcommand{\hmwkAuthorName}{Álvaro González Sotillo} % Your name

%----------------------------------------------------------------------------------------
%	TITLE PAGE
%----------------------------------------------------------------------------------------

\title{
\vspace{2in}
\textmd{\textbf{\hmwkClass:\ \hmwkTitle}}\\
\vspace{0.1in}\large{\textit{\hmwkClassInstructor\ \hmwkClassTime}}
\vspace{3in}
}

\author{\textbf{\hmwkAuthorName}}
\date{} % Insert date here if you want it to appear below your name


%----------------------------------------------------------------------------------------

\usepackage{fancybox}
\newcommand{\codigo}[1]{\texttt{#1}}


% CUADRITO
\newsavebox{\fmboxx}
\newenvironment{cuadrito}[1][14cm]
{\noindent \begin{center} \begin{lrbox}{\fmboxx}\noindent\begin{minipage}{#1}}
{\end{minipage}\end{lrbox}\noindent\shadowbox{\usebox{\fmboxx}} \end{center}}





\newenvironment{entradasalida}[2][14cm]
{
  \newcommand{\elnombredelafiguradeentradasalida}{#2}
  \begin{figure}[h]
    \begin{cuadrito}[#1]
      \begin{scriptsize}
\begin{alltt}
}
{%
\end{alltt}%
      \end{scriptsize}%
    \end{cuadrito}%
    \caption{\elnombredelafiguradeentradasalida}
  \end{figure}
}


\newenvironment{entradasalidacols}[2][14cm]
{
\newcommand{\elnombredelafiguradeentradasalida}{#2}
%\begin{wrapfigure}{}{0.1}
\begin{cuadrito}[#1]
\begin{scriptsize}
\begin{alltt}
}
{
\end{alltt}
\end{scriptsize}
\end{cuadrito}
\captionof{figure}{\elnombredelafiguradeentradasalida}
%\end{wrapfigure}
}


\usepackage{pgffor}
\newcommand{\ficheroautoref}[0]{%
  \foreach \ficherotex in {../../../common/plantilla-ejercicio.tex,../../common/plantilla-ejercicio.tex,../common/plantilla-ejercicio.tex,../apuntes/common/plantilla-ejercicio.tex}{

    \IfFileExists{\ficherotex}%
    {%
      \textattachfile[mimetype=text/plain,%
      description={La plantilla},%
      subject={La plantilla}]%
      {\ficherotex}%
      {}%
      % RECUPERO ESPACIO VERTICAL PERDIDO
      \vspace{-6em}%
    }%
    {}%
  }
  

  \textattachfile[mimetype=text/plain,
  description={El fichero TEX original para crear este documento, no sea que se nos pierda},
  subject={El fichero TEX original para crear este documento, no sea que se nos pierda}]
  {\currfilename}{}

}

\newcommand{\entradausuario}[1]{\textit{\textbf{#1}}}

\newcommand{\enlace}[2]{\textcolor{blue}{{\href{#1}{#2}}}}

\newcommand{\adjuntarfichero}[3]{
\textattachfile[mimetype=text/plain,
color={0 0 0},
description={#3},
subject={#1}]
{#1}
{\textcolor{blue}{\codigo{#2}}}
}

\newcommand{\adjuntardoc}[2]{%
\textattachfile[mimetype=text/plain,%
color={0 0 0},%
description={#1},%
subject={#1}]%
{#1}%
{\textcolor{blue}{#2}}%
}%


\newcommand{\plantilladeclase}[2]{
\adjuntarfichero{#1.java}{#2}{Plantilla para la clase #2}
}

% https://en.wikibooks.org/wiki/LaTeX/Source_Code_Listings#Encoding_issue
\lstset{literate=
  {á}{{\'a}}1 {é}{{\'e}}1 {í}{{\'i}}1 {ó}{{\'o}}1 {ú}{{\'u}}1
  {Á}{{\'A}}1 {É}{{\'E}}1 {Í}{{\'I}}1 {Ó}{{\'O}}1 {Ú}{{\'U}}1
  {à}{{\`a}}1 {è}{{\`e}}1 {ì}{{\`i}}1 {ò}{{\`o}}1 {ù}{{\`u}}1
  {À}{{\`A}}1 {È}{{\'E}}1 {Ì}{{\`I}}1 {Ò}{{\`O}}1 {Ù}{{\`U}}1
  {ä}{{\"a}}1 {ë}{{\"e}}1 {ï}{{\"i}}1 {ö}{{\"o}}1 {ü}{{\"u}}1
  {Ä}{{\"A}}1 {Ë}{{\"E}}1 {Ï}{{\"I}}1 {Ö}{{\"O}}1 {Ü}{{\"U}}1
  {â}{{\^a}}1 {ê}{{\^e}}1 {î}{{\^i}}1 {ô}{{\^o}}1 {û}{{\^u}}1
  {Â}{{\^A}}1 {Ê}{{\^E}}1 {Î}{{\^I}}1 {Ô}{{\^O}}1 {Û}{{\^U}}1
  {œ}{{\oe}}1 {Œ}{{\OE}}1 {æ}{{\ae}}1 {Æ}{{\AE}}1 {ß}{{\ss}}1
  {ű}{{\H{u}}}1 {Ű}{{\H{U}}}1 {ő}{{\H{o}}}1 {Ő}{{\H{O}}}1
  {ç}{{\c c}}1 {Ç}{{\c C}}1 {ø}{{\o}}1 {å}{{\r a}}1 {Å}{{\r A}}1
  {€}{{\euro}}1 {£}{{\pounds}}1 {«}{{\guillemotleft}}1
  {»}{{\guillemotright}}1 {ñ}{{\~n}}1 {Ñ}{{\~N}}1 {¿}{{?`}}1
}

\lstset{
  inputencoding=utf8,
  captionpos=b,
  frame=single,
  basicstyle=\small\ttfamily,
  showstringspaces=false,
  numbers=none,
  numbers=left,
  xleftmargin=2em,
  breaklines=true,
  postbreak=\mbox{\textcolor{red}{$\hookrightarrow$}\space}
}

\renewcommand{\lstlistingname}{Listado}
\captionsetup[lstlisting]{font={footnotesize},labelfont=bf,position=bottom}
\captionsetup[figure]{font={footnotesize},labelfont=bf}
\lstnewenvironment{listadojava}[2]
{
  \lstset{language=Java,caption={#1},label={#2}}
  \noindent\minipage{\linewidth}%
}
{\endminipage}

% LISTADO SHELL
\lstnewenvironment{listadoshell}[2]
{
  \lstset{language=bash,caption={#1},label={#2}}
  \noindent\minipage{\linewidth}%
}
{\endminipage}

% LISTADO TXT
\lstnewenvironment{listadotxt}[2]
{
  \lstset{caption={#1},label={#2}, keywords={}}
  \noindent\minipage{\linewidth}%
}
{\endminipage}

% LISTADO SQL
\lstnewenvironment{listadosql}[2]%
{%
  %(el 1 )caption es #1\\%
  %(el 2) label es #2\\%
  \lstset{language=sql,caption={#1},label={#2}}%
  \noindent\minipage{\linewidth}%
}
{\endminipage}
  






\newcommand{\graficosvg}[3][14cm]{
\begin{figure}[htbp]
\centering
\textattachfile{#2.svg}{
\color{black}
\includesvg[width=#1]{#2}
}
\caption{#3}
\end{figure}
}


\newcommand{\graficosvguml}[3][7cm]{
  \texttt{\graficosvg[#1]{#2}{#3}}
}


\newcommand{\primerapagina}{
\newpage
\ficheroautoref
\tableofcontents
\newpage
}

% CAJAS
\usepackage[skins]{tcolorbox}
\newtcolorbox{Aviso}[1][Aviso]{
  enhanced,
  colback=gray!5!white,
  colframe=gray!75!black,fonttitle=\bfseries,
  colbacktitle=gray!85!black,
  attach boxed title to top left={yshift=-2mm,xshift=2mm},
  title=#1
}


\newcommand{\StudentData}{
  \begin{cuadrito}[1\textwidth]
    \vspace{0.3cm}
    \large{
      \textbf{Apellidos:} \hrulefill \\
      \textbf{Nombre:} \hrulefill \\
      \textbf{Fecha:} \hrulefill \hspace{1cm} \textbf{Grupo:} \hrulefill \\
    }
    %\vspace{0.2cm}
  \end{cuadrito}
}

% TEXTO EN MONOESPACIO PERO QUE SE PARTE EN LINEAS
\usepackage{seqsplit}
\newcommand{\tecnico}[1]{\texttt{\seqsplit{#1}}}

% REEMPLAZAR TEXTO, NO LO USO
\def\replace#1#2#3{%
 \def\tmp##1#2{##1#3\tmp}%
   \tmp#1\stopreplace#2\stopreplace}
\def\stopreplace#1\stopreplace{}

\usepackage{eurosym}

\newcommand{\anio}{2017}



\renewcommand{\hmwkTitle}{Diseño E/R, paso a SQL e inserción de datos}
\renewcommand{\hmwkClass}{Gestión de Bases de Datos}

\usepackage{enumitem}% http://ctan.org/pkg/enumitem
%\setlist[itemize]{parsep=0em}

\begin{document}

% \maketitle

% ----------------------------------------------------------------------------------------
%	TABLE OF CONTENTS
% ----------------------------------------------------------------------------------------

% \setcounter{tocdepth}{1} % Uncomment this line if you don't want subsections listed in the ToC

\primerapagina


\section{Objetivo de la práctica}
Se pretende que el alumno sea capaz de pasar desde las necesidades de datos de una empresa hasta las órdenes SQL necesarias para implementar dicha necesidad en Oracle

Se puede descargar \enlace{https://alvarogonzalezsotillo.github.io/apuntes-clase/gestion-bd-asir1/apuntes/3/BD-3-EjercicioCreacionDeTablas.pdf}{la última versión de la práctica en este enlace}

\section{Modelo de datos}
Una compañía de alquiler de coches desea informatizar su gestión.  Su flota de coches se divide en tres categorías: Functional, Advance y Prestige. Cada categoría puede tener diferentes modelos de coche con diferentes acabados. Un modelo concreto puede pertenecer a varias categorías, si el acabado es distinto. El precio por día del alquiler depende de la categoría asignada al coche.

La categoría no depende directamente del acabado del coche: el departamento comercial asigna la categoría atendiendo a criterios de coste y demanda.

Se desea saber los siguientes datos en el acabado de los coches, entre otros:
\begin{itemize}
\item aire acondicionado, climatizador, navegador, bluetooth, mp3, cambio automático, techo solar
\end{itemize}

De cada coche de la flota se desea saber:
\begin{itemize}
\item Marca, modelo y color
\item Categoría
\item Acabado
\item Matrícula y número de bastidor
\end{itemize}

De cada alquiler de coche se desea saber:
\begin{itemize}
\item Datos del cliente (nombre, apellidos, DNI/NIE/pasaporte, edad,
  género)
\item  Delegación donde se recoge el coche, y delegación donde se
  dejará el coche
\item  Número de Km inicial y final
\item  Fecha de inicio y fin del alquiler
\item  Descuento comercial aplicado
\end{itemize}


\begin{homeworkProblem}[: Realizar el diagrama \textbf{ER} con Oracle \textbf{SQLDeveloper} (2 puntos)]
  En el diagrama deben ser visibles (al menos):
  \begin{itemize}
  \item Los nombres de las entidades
  \item Los atributos de las entidades
  \item Las relaciones entre las entidades
  \end{itemize}

\end{homeworkProblem}


\begin{homeworkProblem}[: Crear las tablas de Oracle utilizando \textbf{SQL} (1 punto)]
  Se entregará un fichero de texto de extensión \texttt{.SQL} con las órdenes \textbf{SQL} de creación de tablas, sus claves primarias, extranjeras, valores por defecto y restricciones.

\end{homeworkProblem}



\begin{homeworkProblem}[: Insertar datos en las tablas (2 puntos)]
  Se insertarán los siguientes datos en las tablas. Los campos no especificados se
  rellenarán a gusto del alumno:

  \begin{enumerate}
  \item  \texttt{Pedro Martínez Martínez} alquiló un Mercedes Clase B verde con
    Matrícula \texttt{1234ABC} el \texttt{1-1-\anio} al \texttt{1-2-\anio}. Tenía cambio automático, y
    realizó 200 km.
  \item  \texttt{Pedro Martínez Martínez} alquiló un Renault Fluenze con Matrícula
    \texttt{1111ABC} el \texttt{1-3-\anio} al \texttt{1-4-\anio}. Tenía navegador y bluetooth, y realizó
    100 km.
  \item  Los precios por día de las categorías Functional, Advance y Prestige
    son de 95\euro, 120\euro y 200\euro, respectivamente

  \item  Los coches con cambio automático o navegador se están clasificando
    como categoría \texttt{Prestige}. Los que tienen bluetooth, en la categoría
    \texttt{Advance}.
  \item  \texttt{Juan Pérez Pérez} alquiló un Renault Kangoo con Matrícula \texttt{4321ABC} el
    \texttt{1-3-\anio} al \texttt{1-4-\anio}. Tenía bluetooth, y se le aplicó un descuento del \texttt{5\%}.
    Recogió el coche en Madrid, y lo devolvió en Sevilla.
  \item  \texttt{Manolo Bombo Bombo} alquiló un Renault Modus el \texttt{1-3-\anio}, y aún no lo
    ha devuelto. Tenía bluetooth, y lo recogió en Barcelona.
  \end{enumerate}

  Para comprobar los datos introducidos, se consultarán las siguientes vistas:
  
  \begin{listadosql}{Vistas a crear}{lst:vistas1}
  V_ALQUILERES(nombrecliente, matricula, marca, modelo, categoria, fechainicio, fechafin, descuento)
  V_ACABADOS(matricula, tienebluetooth, tienenavegador, tienecambioautomatico)
  V_CATEGORIAS(categoria, preciodia)
  \end{listadosql}
\end{homeworkProblem}


\begin{homeworkProblem}[: Modificar las tablas (2 puntos)]

  Tras la puesta en marcha del sistema, la dirección de la empresa necesita modificar las tablas ya existentes para almacenar nueva información:
  
  \begin{itemize}
  \item La empresa se ha dado cuenta que necesita realizar revisiones
    periódicas a los coches. Necesita saber, por cada coche:
    \begin{itemize}
    \item Tipo de revisión (ITV, mantenimiento)
    \item Nº de kilómetros (o fecha máxima) para esa revisión.
    \item Nº de kilómetros y fecha en la que se pasaron las anteriores revisiones.
    \end{itemize}


  \item  Algunos coches se han repintado de otros colores. Se necesita saber qué colores ha tenido un coche, en qué fechas.
  \item  Se desea añadir un acabado para asientos calefactados.
  \end{itemize}
\end{homeworkProblem}

\begin{homeworkProblem}[: Inserción de nuevos datos (2 puntos)]

  \begin{itemize}
  \item El coche con matrícula \texttt{1234ABC} se compró \texttt{rojo} el \texttt{1-1-\anio}. Se repintó a \texttt{verde} el \texttt{1-6-\anio}.
  \item  Los Renault deben realizar un mantenimiento cada 15000 km.
    
  \item \texttt{1111ABC} realizó su primera revisión el \texttt{1-3-\anio}.
  \item \texttt{4321ABC} se compró el \texttt{1-2-\anio}, y lleva dos revisiones, el \texttt{1-4-\anio} y el \texttt{1-10-\anio}.

  \item Los Mercedes realizan un mantenimiento cada 20000 km. 
  \item \texttt{1234ABC}   se compró el \texttt{1-1-\anio}, y se revisó por primera vez el \texttt{1-4-\anio}.
  \item Todos los coches se revisan a los 4 años en la ITV.
  \item El coche con matrícula \texttt{1111ABC} tiene asientos calefactados.
  \end{itemize}

  Para comprobar los nuevos datos introducidos, se consultarán las siguientes vistas:
  
  \begin{listadosql}{Vistas a crear tras las modificaciones}{lst:vistas2}
  V_REVISIONESREALIZADAS(matricula, tiporevision, kilometros, fecha)
  V_PROXIMASREVISIONES(matricula, tiporevision, kilometros, fecha)
  V_ACABADOS2(matricula, tienebluetooth, tienenavegador, tienecambioautomatico, tieneasientoscalefactados)
  V_COLORES(matricula,color,fechainiciocolor)
  \end{listadosql}

\end{homeworkProblem}

\section{Instrucciones de entrega}
\begin{itemize}
\item El ejercicio se realizará y entregará de manera individual.
  \begin{itemize}
  \item Solo se admiten trabajos en pareja, si en clase es necesario compartir ordenador.
  \end{itemize}
\item Entrega tu trabajo en un fichero \texttt{ZIP} con
  \begin{itemize}
  \item \texttt{\textbf{1.modelo-er.pdf}}: La imagen de tu modelo ER  con SQLDeveloper
  \item \texttt{\textbf{2.creacion.sql}}: El script de creación de tablas y vistas
  \item \texttt{\textbf{3.insercion.sql}}: El script de inserción de datos en las tablas
  \item \texttt{\textbf{4.modificacion.sql}}: El script de modificación de las tablas y creación de vistas
  \item \texttt{\textbf{5.insercion.sql}}: El script de inserción de datos en las tablas modificadas
  \end{itemize}
\item Sube el documento a \enlace{https://aulavirtual3.educa.madrid.org/ies.alonsodeavellan.alcala/course/view.php?id=189}{la tarea correspondiente en el aula virtual}
\item Presta atención al plazo de entrega (con fecha y hora).
  
\end{itemize}
\end{document}




