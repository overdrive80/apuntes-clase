%%%%%%%%%%%%%%%%%%%%%%%%%%%%%%%%%%%%%%%%%
% Programming/Coding Assignment
% LaTeX Template
%
% This template has been downloaded from:
% http://www.latextemplates.com
%
% Original author:
% Ted Pavlic (http://www.tedpavlic.com)
%
% Note:
% The \lipsum[#] commands throughout this template generate dummy text
% to fill the template out. These commands should all be removed when 
% writing assignment content.
%
% This template uses a Perl script as an example snippet of code, most other
% languages are also usable. Configure them in the "CODE INCLUSION 
% CONFIGURATION" section.
%
%%%%%%%%%%%%%%%%%%%%%%%%%%%%%%%%%%%%%%%%%

%----------------------------------------------------------------------------------------
%	PACKAGES AND OTHER DOCUMENT CONFIGURATIONS
%----------------------------------------------------------------------------------------

\documentclass[a4paper]{article}
\usepackage[utf8]{inputenc}
\usepackage{listingsutf8}
\usepackage[spanish]{babel}




\usepackage{fancyhdr} % Required for custom headers
\usepackage{lastpage} % Required to determine the last page for the footer
\usepackage{extramarks} % Required for headers and footers
\usepackage[usenames,dvipsnames]{color} % Required for custom colors
\usepackage{graphicx} % Required to insert images
\usepackage{listings} % Required for insertion of code
\usepackage{courier} % Required for the courier font
\usepackage{lipsum} % Used for inserting dummy 'Lorem ipsum' text into the template
\usepackage{svg}
\usepackage{attachfile}
\usepackage{currfile}
\usepackage{multicol}
\usepackage{alltt}
\usepackage{framed}
\usepackage{caption}
\usepackage{seqsplit}

\hypersetup{
colorlinks,
citecolor=black,
filecolor=black,
linkcolor=black,
urlcolor=blue
}


% Margins
\topmargin=-0.45in
\evensidemargin=0in
\oddsidemargin=0in
\textwidth=6.5in
\textheight=9.8in
\headsep=0.25in

\linespread{1.1} % Line spacing
\setlength{\parindent}{1.5em}
\setlength{\parskip}{1em}


% Set up the header and footer
\pagestyle{fancy}
%\lhead{\hmwkAuthorName} % Top left header
\lhead{\hmwkClass}
\rhead{\hmwkTitle} % Top right header
\chead{} % Top center head
\lfoot{\lastxmark} % Bottom left footer
\cfoot{} % Bottom center footer
\rfoot{ \thepage\ / \protect\pageref{LastPage}} % Bottom right footer
\renewcommand\headrulewidth{0.4pt} % Size of the header rule
\renewcommand\footrulewidth{0.4pt} % Size of the footer rule



%----------------------------------------------------------------------------------------
%	CODE INCLUSION CONFIGURATION
%----------------------------------------------------------------------------------------
\renewcommand{\ttdefault}{pcr}
\definecolor{MyDarkGreen}{rgb}{0.0,0.4,0.0} % This is the color used for comments
\lstloadlanguages{Perl} % Load Perl syntax for listings, for a list of other languages supported see: ftp://ftp.tex.ac.uk/tex-archive/macros/latex/contrib/listings/listings.pdf
\lstset{
language=sh, % Use Perl in this example
frame=single, % Single frame around code
basicstyle=\small\ttfamily, % Use small true type font
keywordstyle=[1]\small\color{Blue}\bf, % Perl functions bold and blue
keywordstyle=[2]\small\color{Purple}, % Perl function arguments purple
keywordstyle=[3]\small\color{Blue}\underbar, % Custom functions underlined and blue
identifierstyle=, % Nothing special about identifiers                                         
commentstyle=\small\color{MyDarkGreen}, % Comments small dark green courier font
stringstyle=\small\color{Purple}, % Strings are purple
showstringspaces=false, % Don't put marks in string spaces
tabsize=5, % 5 spaces per tab
morecomment=[l][\small\color{Blue}]{...}, % Line continuation (...) like blue comment
numbers=none, % Line numbers on left
firstnumber=1, % Line numbers start with line 1
numberstyle=\tiny\color{Blue}, % Line numbers are blue and small
stepnumber=0 % Line numbers go in steps of 5
}

% Creates a new command to include a perl script, the first parameter is the filename of the script (without .pl), the second parameter is the caption
\newcommand{\perlscript}[2]{
\begin{itemize}
\item[]\lstinputlisting[caption=#2,label=#1]{#1.pl}
\end{itemize}
}

%----------------------------------------------------------------------------------------
%	DOCUMENT STRUCTURE COMMANDS
%	Skip this unless you know what you're doing
%----------------------------------------------------------------------------------------

% Header and footer for when a page split occurs within a problem environment
\newcommand{\enterProblemHeader}[1]{
%\nobreak\extramarks{#1}{#1 continued on next page\ldots}\nobreak
%\nobreak\extramarks{#1 (continued)}{#1 continued on next page\ldots}\nobreak
}

% Header and footer for when a page split occurs between problem environments
\newcommand{\exitProblemHeader}[1]{
%\nobreak\extramarks{#1 (continued)}{#1 continued on next page\ldots}\nobreak
%\nobreak\extramarks{#1}{}\nobreak
}

\setcounter{secnumdepth}{0} % Removes default section numbers
\newcounter{homeworkProblemCounter} % Creates a counter to keep track of the number of problems

\newcommand{\homeworkProblemName}{}
\newenvironment{homeworkProblem}[1][]{ % Makes a new environment called homeworkProblem which takes 1 argument (custom name) but the default is "Problem #"
\stepcounter{homeworkProblemCounter} % Increase counter for number of problems
\renewcommand{\homeworkProblemName}{Ejercicio \arabic{homeworkProblemCounter} #1} % Assign \homeworkProblemName the name of the problem
\section{\homeworkProblemName} % Make a section in the document with the custom problem count
\enterProblemHeader{\homeworkProblemName} % Header and footer within the environment
}{
\exitProblemHeader{\homeworkProblemName} % Header and footer after the environment
%\clearpage
}


\newcommand{\problemAnswer}[1]{ % Defines the problem answer command with the content as the only argument
\noindent\framebox[\columnwidth][c]{\begin{minipage}{0.98\columnwidth}#1\end{minipage}} % Makes the box around the problem answer and puts the content inside
}

\newcommand{\homeworkSectionName}{}
\newenvironment{homeworkSection}[1]{ % New environment for sections within homework problems, takes 1 argument - the name of the section
\renewcommand{\homeworkSectionName}{#1} % Assign \homeworkSectionName to the name of the section from the environment argument
\subsection{\homeworkSectionName} % Make a subsection with the custom name of the subsection
\enterProblemHeader{\homeworkProblemName\ [\homeworkSectionName]} % Header and footer within the environment
}{
\enterProblemHeader{\homeworkProblemName} % Header and footer after the environment
}

%----------------------------------------------------------------------------------------
%	NAME AND CLASS SECTION
%----------------------------------------------------------------------------------------

\newcommand{\hmwkTitle}{Redefine el hmwkTitle} % Assignment title
\newcommand{\hmwkDueDate}{asdfadsf} % Due date
\newcommand{\hmwkClass}{Redefine hmwkClass} % Course/class
\newcommand{\hmwkClassTime}{} % Class/lecture time
\newcommand{\hmwkClassInstructor}{} % Teacher/lecturer
\newcommand{\hmwkAuthorName}{Álvaro González Sotillo} % Your name

%----------------------------------------------------------------------------------------
%	TITLE PAGE
%----------------------------------------------------------------------------------------

\title{
\vspace{2in}
\textmd{\textbf{\hmwkClass:\ \hmwkTitle}}\\
\vspace{0.1in}\large{\textit{\hmwkClassInstructor\ \hmwkClassTime}}
\vspace{3in}
}

\author{\textbf{\hmwkAuthorName}}
\date{} % Insert date here if you want it to appear below your name


%----------------------------------------------------------------------------------------

\usepackage{fancybox}
\newcommand{\codigo}[1]{\texttt{#1}}


% CUADRITO
\newsavebox{\fmboxx}
\newenvironment{cuadrito}[1][14cm]
{\noindent \begin{center} \begin{lrbox}{\fmboxx}\noindent\begin{minipage}{#1}}
{\end{minipage}\end{lrbox}\noindent\shadowbox{\usebox{\fmboxx}} \end{center}}





\newenvironment{entradasalida}[2][14cm]
{
  \newcommand{\elnombredelafiguradeentradasalida}{#2}
  \begin{figure}[h]
    \begin{cuadrito}[#1]
      \begin{scriptsize}
\begin{alltt}
}
{%
\end{alltt}%
      \end{scriptsize}%
    \end{cuadrito}%
    \caption{\elnombredelafiguradeentradasalida}
  \end{figure}
}


\newenvironment{entradasalidacols}[2][14cm]
{
\newcommand{\elnombredelafiguradeentradasalida}{#2}
%\begin{wrapfigure}{}{0.1}
\begin{cuadrito}[#1]
\begin{scriptsize}
\begin{alltt}
}
{
\end{alltt}
\end{scriptsize}
\end{cuadrito}
\captionof{figure}{\elnombredelafiguradeentradasalida}
%\end{wrapfigure}
}


\usepackage{pgffor}
\newcommand{\ficheroautoref}[0]{%
  \foreach \ficherotex in {../../../common/plantilla-ejercicio.tex,../../common/plantilla-ejercicio.tex,../common/plantilla-ejercicio.tex,../apuntes/common/plantilla-ejercicio.tex}{

    \IfFileExists{\ficherotex}%
    {%
      \textattachfile[mimetype=text/plain,%
      description={La plantilla},%
      subject={La plantilla}]%
      {\ficherotex}%
      {}%
      % RECUPERO ESPACIO VERTICAL PERDIDO
      \vspace{-6em}%
    }%
    {}%
  }
  

  \textattachfile[mimetype=text/plain,
  description={El fichero TEX original para crear este documento, no sea que se nos pierda},
  subject={El fichero TEX original para crear este documento, no sea que se nos pierda}]
  {\currfilename}{}

}

\newcommand{\entradausuario}[1]{\textit{\textbf{#1}}}

\newcommand{\enlace}[2]{\textcolor{blue}{{\href{#1}{#2}}}}

\newcommand{\adjuntarfichero}[3]{
\textattachfile[mimetype=text/plain,
color={0 0 0},
description={#3},
subject={#1}]
{#1}
{\textcolor{blue}{\codigo{#2}}}
}

\newcommand{\adjuntardoc}[2]{%
\textattachfile[mimetype=text/plain,%
color={0 0 0},%
description={#1},%
subject={#1}]%
{#1}%
{\textcolor{blue}{#2}}%
}%


\newcommand{\plantilladeclase}[2]{
\adjuntarfichero{#1.java}{#2}{Plantilla para la clase #2}
}

% https://en.wikibooks.org/wiki/LaTeX/Source_Code_Listings#Encoding_issue
\lstset{literate=
  {á}{{\'a}}1 {é}{{\'e}}1 {í}{{\'i}}1 {ó}{{\'o}}1 {ú}{{\'u}}1
  {Á}{{\'A}}1 {É}{{\'E}}1 {Í}{{\'I}}1 {Ó}{{\'O}}1 {Ú}{{\'U}}1
  {à}{{\`a}}1 {è}{{\`e}}1 {ì}{{\`i}}1 {ò}{{\`o}}1 {ù}{{\`u}}1
  {À}{{\`A}}1 {È}{{\'E}}1 {Ì}{{\`I}}1 {Ò}{{\`O}}1 {Ù}{{\`U}}1
  {ä}{{\"a}}1 {ë}{{\"e}}1 {ï}{{\"i}}1 {ö}{{\"o}}1 {ü}{{\"u}}1
  {Ä}{{\"A}}1 {Ë}{{\"E}}1 {Ï}{{\"I}}1 {Ö}{{\"O}}1 {Ü}{{\"U}}1
  {â}{{\^a}}1 {ê}{{\^e}}1 {î}{{\^i}}1 {ô}{{\^o}}1 {û}{{\^u}}1
  {Â}{{\^A}}1 {Ê}{{\^E}}1 {Î}{{\^I}}1 {Ô}{{\^O}}1 {Û}{{\^U}}1
  {œ}{{\oe}}1 {Œ}{{\OE}}1 {æ}{{\ae}}1 {Æ}{{\AE}}1 {ß}{{\ss}}1
  {ű}{{\H{u}}}1 {Ű}{{\H{U}}}1 {ő}{{\H{o}}}1 {Ő}{{\H{O}}}1
  {ç}{{\c c}}1 {Ç}{{\c C}}1 {ø}{{\o}}1 {å}{{\r a}}1 {Å}{{\r A}}1
  {€}{{\euro}}1 {£}{{\pounds}}1 {«}{{\guillemotleft}}1
  {»}{{\guillemotright}}1 {ñ}{{\~n}}1 {Ñ}{{\~N}}1 {¿}{{?`}}1
}

\lstset{
  inputencoding=utf8,
  captionpos=b,
  frame=single,
  basicstyle=\small\ttfamily,
  showstringspaces=false,
  numbers=none,
  numbers=left,
  xleftmargin=2em,
  breaklines=true,
  postbreak=\mbox{\textcolor{red}{$\hookrightarrow$}\space}
}

\renewcommand{\lstlistingname}{Listado}
\captionsetup[lstlisting]{font={footnotesize},labelfont=bf,position=bottom}
\captionsetup[figure]{font={footnotesize},labelfont=bf}
\lstnewenvironment{listadojava}[2]
{
  \lstset{language=Java,caption={#1},label={#2}}
  \noindent\minipage{\linewidth}%
}
{\endminipage}

% LISTADO SHELL
\lstnewenvironment{listadoshell}[2]
{
  \lstset{language=bash,caption={#1},label={#2}}
  \noindent\minipage{\linewidth}%
}
{\endminipage}

% LISTADO TXT
\lstnewenvironment{listadotxt}[2]
{
  \lstset{caption={#1},label={#2}, keywords={}}
  \noindent\minipage{\linewidth}%
}
{\endminipage}

% LISTADO SQL
\lstnewenvironment{listadosql}[2]%
{%
  %(el 1 )caption es #1\\%
  %(el 2) label es #2\\%
  \lstset{language=sql,caption={#1},label={#2}}%
  \noindent\minipage{\linewidth}%
}
{\endminipage}
  






\newcommand{\graficosvg}[3][14cm]{
\begin{figure}[htbp]
\centering
\textattachfile{#2.svg}{
\color{black}
\includesvg[width=#1]{#2}
}
\caption{#3}
\end{figure}
}


\newcommand{\graficosvguml}[3][7cm]{
  \texttt{\graficosvg[#1]{#2}{#3}}
}


\newcommand{\primerapagina}{
\newpage
\ficheroautoref
\tableofcontents
\newpage
}

% CAJAS
\usepackage[skins]{tcolorbox}
\newtcolorbox{Aviso}[1][Aviso]{
  enhanced,
  colback=gray!5!white,
  colframe=gray!75!black,fonttitle=\bfseries,
  colbacktitle=gray!85!black,
  attach boxed title to top left={yshift=-2mm,xshift=2mm},
  title=#1
}


\newcommand{\StudentData}{
  \begin{cuadrito}[1\textwidth]
    \vspace{0.3cm}
    \large{
      \textbf{Apellidos:} \hrulefill \\
      \textbf{Nombre:} \hrulefill \\
      \textbf{Fecha:} \hrulefill \hspace{1cm} \textbf{Grupo:} \hrulefill \\
    }
    %\vspace{0.2cm}
  \end{cuadrito}
}

% TEXTO EN MONOESPACIO PERO QUE SE PARTE EN LINEAS
\usepackage{seqsplit}
\newcommand{\tecnico}[1]{\texttt{\seqsplit{#1}}}

% REEMPLAZAR TEXTO, NO LO USO
\def\replace#1#2#3{%
 \def\tmp##1#2{##1#3\tmp}%
   \tmp#1\stopreplace#2\stopreplace}
\def\stopreplace#1\stopreplace{}

\usepackage{eurosym}





\renewcommand{\hmwkTitle}{Usuarios y permisos}
\renewcommand{\hmwkClass}{ASGBD}

\usepackage{blindtext}

\begin{document}

% \maketitle

% ----------------------------------------------------------------------------------------
%	TABLE OF CONTENTS
% ----------------------------------------------------------------------------------------

% \setcounter{tocdepth}{1} % Uncomment this line if you don't want subsections listed in the ToC

\primerapagina

\section{Objetivo de la práctica}
En esta práctica utilizaremos a la base de datos \textbf{Oracle} como un verdadero servidor, conectándonos desde otros ordenadores. Para ello:
\begin{itemize}
\item Necesitamos poner accesible por red nuestro servidor
\item Crearemos tablas en un \textit{tablespace} separado
\item Crearemos un usuario para cada compañero de clase
\item Asignaremos privilegios utilizando un rol
\end{itemize}

\begin{homeworkProblem}[: Poner Oracle disponible por red]

  Los demás compañeros deben poder acceder a nuestro servidor de \textbf{Oracle}. Para ello
  \begin{enumerate}
  \item La máquina virtual debe ser accedida desde el resto del aula. El tipo de conexión será \textit{bridged}
  \item \textbf{Centos} tiene activado un firewall. Hay que desactivarlo como se indica en
    \begin{itemize}
    \item  \url{https://www.liquidweb.com/kb/how-to-stop-and-disable-firewalld-on-centos-7/}
    \end{itemize}
    
  \item La dirección IP se asigna actualmente por DHCP. Esto es un inconveniente porque puede variar cada día. Es mejor utilizar un nombre, así que instalaremos \textbf{avahi}
    \begin{itemize}
    \item \url{https://en.wikipedia.org/wiki/Avahi\_(software)}
    \item Instalaremos repositorios de \textit{software} adicionales con \texttt{sudo yum install epel-release}
    \item Después, instalaremos \textbf{avahi} con \texttt{sudo yum install avahi avahi-tools nss-mdns}

    \end{itemize}

  \item Tu ordenador será accesible con el nombre \texttt{nombre-de-host.local}.
  \item Pide al profesor que añada el nombre de tu ordenador en la siguiente hoja de cálculo: \url{https://goo.gl/FnMmrM}
    
    
  \end{enumerate}

  Cuando tengas estos cambios, pide al profesor que compruebe que funcionan.
\end{homeworkProblem}


\begin{homeworkProblem}[: Crear usuarios para tus compañeros]

  Crea un usuario para tí, uno para cada uno de tus compañeros, y uno para el profesor. La contraseña inicial será la misma que el nombre, excepto en tu usuario que debería ser una contraseña secreta:
  \begin{multicols}{3}
    \begin{itemize}
    \item ANGEL
    \item BANEGAS
    \item BERMUDEZ
    \item BROWN
    \item CABRERA
    \item CALATRAVA
    \item DELMASTRO
    \item DZIERZAK
    \item EXTREMO
    \item FERNANDEZ
    \item FRONTELO
    \item GARCIA
    \item GUINEA
    \item IRUELA
    \item LOZANO
    \item LUENGO
    \item MONTERO
    \item MORALES
    \item PEREZ
    \item RIVERO
    \item RODRIGUEZ
    \item ROMERO
    \item ROSA
    \item SANTAREN
    \item UTRERO
    \item PROFESOR
    \end{itemize}
  \end{multicols}
  {\small \textbf{Nota}: Son muchos usuarios, así que es aconsejable utilizar un \textit{script} en vez de crearlos manualmente.}

  Los usuarios necesitarán el rol \texttt{connect}. Cuando tengas listos los usuarios:
  \begin{itemize}
    \item El resto de usuarios podrán conectarse con \texttt{sqlplus USUARIO/USUARIO@HOST:1521/asir}
    \item Pide al profesor que compruebe su usuario
    \item Pide a algunos compañeros que comprueben su usuario
  \end{itemize}
\end{homeworkProblem}

\begin{homeworkProblem}[: Creación de las tablas]
  Cambia la contraseña de tu usuario, si no lo has hecho ya, para que ningún compañero pueda utilizarlo (con \texttt{ALTER USER}).

  \begin{enumerate}
  \item Crea un \textit{tablespace} de nombre \texttt{multas}
  \item Con tu propio usuario, crea las tablas en ese \textit{tablespace}
    \begin{itemize}
    \item Utiliza el \enlace{https://alvarogonzalezsotillo.github.io/sistemas-gestores-bbdd-asir2/apuntes/2/sql/multas.sql}{\textit{script} \texttt{multas.sql} para la creación de las tablas}
    \item Tendrás que modificar el \textit{script} para que tenga en cuenta tu \textit{tablespace}
    \item Tu usuario deberá tener cuota en el \texttt{tablespace}
    \end{itemize}
  \end{enumerate}
\end{homeworkProblem}

\begin{homeworkProblem}[: Dar acceso a otros usuarios a un campo de tus tablas]
  Tras la importación, haz que las tablas puedan ser leidas por el resto de usuarios:
  \begin{itemize}
  \item Haz que el resto de usuarios pueda realizar \texttt{SELECT} sobre tus tablas.  
  \item Crea sinónimos en todos los usuarios para que puedan acceder a tus tablas sin problemas
    \begin{itemize}
    \item Por ejemplo, el usuario \texttt{profesor} debería poder ejecutar \texttt{SELECT * FROM MULTAS}, puesto que habrás creado un sinónimo del tipo \texttt{CREATE PUBLIC SYNONYM MULTAS FOR \textit{MIUSUARIO}.MULTAS}.
    \end{itemize}
  \end{itemize}

  Después, haz que puedan escribir en un campo de la tabla \texttt{MULTAS}:
  \begin{itemize}
  \item Crea un nuevo campo en la tabla \texttt{MULTAS}: \texttt{PAGADO}, de tipo \texttt{NUMBER(1,0)}. Tendrá el valor \texttt{1} si la multa ha sido pagada, y \texttt{0} en caso contrario (cláusula \texttt{DEFAULT} y \texttt{CHECK})
    
  \item Da permisos al resto de usuarios para poder escribir en este campo, con una orden \texttt{GRANT}
    \begin{itemize}
    \item \enlace{http://stackoverflow.com/questions/14462353/grant-alter-on-only-one-column-in-table}{http://stackoverflow.com/questions/14462353/grant-alter-on-only-one-column-in-table}
    \end{itemize}

  \item Pide a algún otro compañero que compruebe que funciona
  \item Pide al profesor que compruebe que funciona.

  \end{itemize}
  

\end{homeworkProblem}


\section{Instrucciones de entrega}
\begin{itemize}
\item Describe en una memoria todos los pasos que has necesitado para realizar la práctica. Incluye también los errores y como los has solucionado.
\item El ejercicio se realizará y entregará de manera individual.
  \begin{itemize}
  \item Solo se admiten trabajos en pareja, si en clase es necesario compartir ordenador.
  \end{itemize}
\item El profesor debe comprobar que tu base de datos funciona en clase. Pídele que lo revise cuando lo tengas listo.
\item Entrega tu trabajo en formato \textbf{doc}, \textbf{docx}, \textbf{odt} o \textbf{pdf}.
\item También puede entregarse como una entrada de blog. Para ello, sube un archivo con la URL de la entrada.
\item Sube el documento a la tarea correspondiente \enlace{https://aulavirtual3.educa.madrid.org/ies.alonsodeavellan.alcala/course/view.php?id=189}{en el aula virtual}
\item Presta atención al plazo de entrega (con fecha y hora).
  
\end{itemize}
\end{document}
