%%%%%%%%%%%%%%%%%%%%%%%%%%%%%%%%%%%%%%%%%
% Programming/Coding Assignment
% LaTeX Template
%
% This template has been downloaded from:
% http://www.latextemplates.com
%
% Original author:
% Ted Pavlic (http://www.tedpavlic.com)
%
% Note:
% The \lipsum[#] commands throughout this template generate dummy text
% to fill the template out. These commands should all be removed when 
% writing assignment content.
%
% This template uses a Perl script as an example snippet of code, most other
% languages are also usable. Configure them in the "CODE INCLUSION 
% CONFIGURATION" section.
%
%%%%%%%%%%%%%%%%%%%%%%%%%%%%%%%%%%%%%%%%%

%----------------------------------------------------------------------------------------
%	PACKAGES AND OTHER DOCUMENT CONFIGURATIONS
%----------------------------------------------------------------------------------------

\documentclass[a4paper]{article}
\usepackage[utf8]{inputenc}
\usepackage{listingsutf8}
\usepackage[spanish]{babel}




\usepackage{fancyhdr} % Required for custom headers
\usepackage{lastpage} % Required to determine the last page for the footer
\usepackage{extramarks} % Required for headers and footers
\usepackage[usenames,dvipsnames]{color} % Required for custom colors
\usepackage{graphicx} % Required to insert images
\usepackage{listings} % Required for insertion of code
\usepackage{courier} % Required for the courier font
\usepackage{lipsum} % Used for inserting dummy 'Lorem ipsum' text into the template
\usepackage{svg}
\usepackage{attachfile}
\usepackage{currfile}
\usepackage{multicol}
\usepackage{alltt}
\usepackage{framed}
\usepackage{caption}
\usepackage{seqsplit}

\hypersetup{
colorlinks,
citecolor=black,
filecolor=black,
linkcolor=black,
urlcolor=blue
}


% Margins
\topmargin=-0.45in
\evensidemargin=0in
\oddsidemargin=0in
\textwidth=6.5in
\textheight=9.8in
\headsep=0.25in

\linespread{1.1} % Line spacing
\setlength{\parindent}{1.5em}
\setlength{\parskip}{1em}


% Set up the header and footer
\pagestyle{fancy}
%\lhead{\hmwkAuthorName} % Top left header
\lhead{\hmwkClass}
\rhead{\hmwkTitle} % Top right header
\chead{} % Top center head
\lfoot{\lastxmark} % Bottom left footer
\cfoot{} % Bottom center footer
\rfoot{ \thepage\ / \protect\pageref{LastPage}} % Bottom right footer
\renewcommand\headrulewidth{0.4pt} % Size of the header rule
\renewcommand\footrulewidth{0.4pt} % Size of the footer rule



%----------------------------------------------------------------------------------------
%	CODE INCLUSION CONFIGURATION
%----------------------------------------------------------------------------------------
\renewcommand{\ttdefault}{pcr}
\definecolor{MyDarkGreen}{rgb}{0.0,0.4,0.0} % This is the color used for comments
\lstloadlanguages{Perl} % Load Perl syntax for listings, for a list of other languages supported see: ftp://ftp.tex.ac.uk/tex-archive/macros/latex/contrib/listings/listings.pdf
\lstset{
language=sh, % Use Perl in this example
frame=single, % Single frame around code
basicstyle=\small\ttfamily, % Use small true type font
keywordstyle=[1]\small\color{Blue}\bf, % Perl functions bold and blue
keywordstyle=[2]\small\color{Purple}, % Perl function arguments purple
keywordstyle=[3]\small\color{Blue}\underbar, % Custom functions underlined and blue
identifierstyle=, % Nothing special about identifiers                                         
commentstyle=\small\color{MyDarkGreen}, % Comments small dark green courier font
stringstyle=\small\color{Purple}, % Strings are purple
showstringspaces=false, % Don't put marks in string spaces
tabsize=5, % 5 spaces per tab
morecomment=[l][\small\color{Blue}]{...}, % Line continuation (...) like blue comment
numbers=none, % Line numbers on left
firstnumber=1, % Line numbers start with line 1
numberstyle=\tiny\color{Blue}, % Line numbers are blue and small
stepnumber=0 % Line numbers go in steps of 5
}

% Creates a new command to include a perl script, the first parameter is the filename of the script (without .pl), the second parameter is the caption
\newcommand{\perlscript}[2]{
\begin{itemize}
\item[]\lstinputlisting[caption=#2,label=#1]{#1.pl}
\end{itemize}
}

%----------------------------------------------------------------------------------------
%	DOCUMENT STRUCTURE COMMANDS
%	Skip this unless you know what you're doing
%----------------------------------------------------------------------------------------

% Header and footer for when a page split occurs within a problem environment
\newcommand{\enterProblemHeader}[1]{
%\nobreak\extramarks{#1}{#1 continued on next page\ldots}\nobreak
%\nobreak\extramarks{#1 (continued)}{#1 continued on next page\ldots}\nobreak
}

% Header and footer for when a page split occurs between problem environments
\newcommand{\exitProblemHeader}[1]{
%\nobreak\extramarks{#1 (continued)}{#1 continued on next page\ldots}\nobreak
%\nobreak\extramarks{#1}{}\nobreak
}

\setcounter{secnumdepth}{0} % Removes default section numbers
\newcounter{homeworkProblemCounter} % Creates a counter to keep track of the number of problems

\newcommand{\homeworkProblemName}{}
\newenvironment{homeworkProblem}[1][]{ % Makes a new environment called homeworkProblem which takes 1 argument (custom name) but the default is "Problem #"
\stepcounter{homeworkProblemCounter} % Increase counter for number of problems
\renewcommand{\homeworkProblemName}{Ejercicio \arabic{homeworkProblemCounter} #1} % Assign \homeworkProblemName the name of the problem
\section{\homeworkProblemName} % Make a section in the document with the custom problem count
\enterProblemHeader{\homeworkProblemName} % Header and footer within the environment
}{
\exitProblemHeader{\homeworkProblemName} % Header and footer after the environment
%\clearpage
}


\newcommand{\problemAnswer}[1]{ % Defines the problem answer command with the content as the only argument
\noindent\framebox[\columnwidth][c]{\begin{minipage}{0.98\columnwidth}#1\end{minipage}} % Makes the box around the problem answer and puts the content inside
}

\newcommand{\homeworkSectionName}{}
\newenvironment{homeworkSection}[1]{ % New environment for sections within homework problems, takes 1 argument - the name of the section
\renewcommand{\homeworkSectionName}{#1} % Assign \homeworkSectionName to the name of the section from the environment argument
\subsection{\homeworkSectionName} % Make a subsection with the custom name of the subsection
\enterProblemHeader{\homeworkProblemName\ [\homeworkSectionName]} % Header and footer within the environment
}{
\enterProblemHeader{\homeworkProblemName} % Header and footer after the environment
}

%----------------------------------------------------------------------------------------
%	NAME AND CLASS SECTION
%----------------------------------------------------------------------------------------

\newcommand{\hmwkTitle}{Redefine el hmwkTitle} % Assignment title
\newcommand{\hmwkDueDate}{asdfadsf} % Due date
\newcommand{\hmwkClass}{Redefine hmwkClass} % Course/class
\newcommand{\hmwkClassTime}{} % Class/lecture time
\newcommand{\hmwkClassInstructor}{} % Teacher/lecturer
\newcommand{\hmwkAuthorName}{Álvaro González Sotillo} % Your name

%----------------------------------------------------------------------------------------
%	TITLE PAGE
%----------------------------------------------------------------------------------------

\title{
\vspace{2in}
\textmd{\textbf{\hmwkClass:\ \hmwkTitle}}\\
\vspace{0.1in}\large{\textit{\hmwkClassInstructor\ \hmwkClassTime}}
\vspace{3in}
}

\author{\textbf{\hmwkAuthorName}}
\date{} % Insert date here if you want it to appear below your name


%----------------------------------------------------------------------------------------

\usepackage{fancybox}
\newcommand{\codigo}[1]{\texttt{#1}}


% CUADRITO
\newsavebox{\fmboxx}
\newenvironment{cuadrito}[1][14cm]
{\noindent \begin{center} \begin{lrbox}{\fmboxx}\noindent\begin{minipage}{#1}}
{\end{minipage}\end{lrbox}\noindent\shadowbox{\usebox{\fmboxx}} \end{center}}





\newenvironment{entradasalida}[2][14cm]
{
  \newcommand{\elnombredelafiguradeentradasalida}{#2}
  \begin{figure}[h]
    \begin{cuadrito}[#1]
      \begin{scriptsize}
\begin{alltt}
}
{%
\end{alltt}%
      \end{scriptsize}%
    \end{cuadrito}%
    \caption{\elnombredelafiguradeentradasalida}
  \end{figure}
}


\newenvironment{entradasalidacols}[2][14cm]
{
\newcommand{\elnombredelafiguradeentradasalida}{#2}
%\begin{wrapfigure}{}{0.1}
\begin{cuadrito}[#1]
\begin{scriptsize}
\begin{alltt}
}
{
\end{alltt}
\end{scriptsize}
\end{cuadrito}
\captionof{figure}{\elnombredelafiguradeentradasalida}
%\end{wrapfigure}
}


\usepackage{pgffor}
\newcommand{\ficheroautoref}[0]{%
  \foreach \ficherotex in {../../../common/plantilla-ejercicio.tex,../../common/plantilla-ejercicio.tex,../common/plantilla-ejercicio.tex,../apuntes/common/plantilla-ejercicio.tex}{

    \IfFileExists{\ficherotex}%
    {%
      \textattachfile[mimetype=text/plain,%
      description={La plantilla},%
      subject={La plantilla}]%
      {\ficherotex}%
      {}%
      % RECUPERO ESPACIO VERTICAL PERDIDO
      \vspace{-6em}%
    }%
    {}%
  }
  

  \textattachfile[mimetype=text/plain,
  description={El fichero TEX original para crear este documento, no sea que se nos pierda},
  subject={El fichero TEX original para crear este documento, no sea que se nos pierda}]
  {\currfilename}{}

}

\newcommand{\entradausuario}[1]{\textit{\textbf{#1}}}

\newcommand{\enlace}[2]{\textcolor{blue}{{\href{#1}{#2}}}}

\newcommand{\adjuntarfichero}[3]{
\textattachfile[mimetype=text/plain,
color={0 0 0},
description={#3},
subject={#1}]
{#1}
{\textcolor{blue}{\codigo{#2}}}
}

\newcommand{\adjuntardoc}[2]{%
\textattachfile[mimetype=text/plain,%
color={0 0 0},%
description={#1},%
subject={#1}]%
{#1}%
{\textcolor{blue}{#2}}%
}%


\newcommand{\plantilladeclase}[2]{
\adjuntarfichero{#1.java}{#2}{Plantilla para la clase #2}
}

% https://en.wikibooks.org/wiki/LaTeX/Source_Code_Listings#Encoding_issue
\lstset{literate=
  {á}{{\'a}}1 {é}{{\'e}}1 {í}{{\'i}}1 {ó}{{\'o}}1 {ú}{{\'u}}1
  {Á}{{\'A}}1 {É}{{\'E}}1 {Í}{{\'I}}1 {Ó}{{\'O}}1 {Ú}{{\'U}}1
  {à}{{\`a}}1 {è}{{\`e}}1 {ì}{{\`i}}1 {ò}{{\`o}}1 {ù}{{\`u}}1
  {À}{{\`A}}1 {È}{{\'E}}1 {Ì}{{\`I}}1 {Ò}{{\`O}}1 {Ù}{{\`U}}1
  {ä}{{\"a}}1 {ë}{{\"e}}1 {ï}{{\"i}}1 {ö}{{\"o}}1 {ü}{{\"u}}1
  {Ä}{{\"A}}1 {Ë}{{\"E}}1 {Ï}{{\"I}}1 {Ö}{{\"O}}1 {Ü}{{\"U}}1
  {â}{{\^a}}1 {ê}{{\^e}}1 {î}{{\^i}}1 {ô}{{\^o}}1 {û}{{\^u}}1
  {Â}{{\^A}}1 {Ê}{{\^E}}1 {Î}{{\^I}}1 {Ô}{{\^O}}1 {Û}{{\^U}}1
  {œ}{{\oe}}1 {Œ}{{\OE}}1 {æ}{{\ae}}1 {Æ}{{\AE}}1 {ß}{{\ss}}1
  {ű}{{\H{u}}}1 {Ű}{{\H{U}}}1 {ő}{{\H{o}}}1 {Ő}{{\H{O}}}1
  {ç}{{\c c}}1 {Ç}{{\c C}}1 {ø}{{\o}}1 {å}{{\r a}}1 {Å}{{\r A}}1
  {€}{{\euro}}1 {£}{{\pounds}}1 {«}{{\guillemotleft}}1
  {»}{{\guillemotright}}1 {ñ}{{\~n}}1 {Ñ}{{\~N}}1 {¿}{{?`}}1
}

\lstset{
  inputencoding=utf8,
  captionpos=b,
  frame=single,
  basicstyle=\small\ttfamily,
  showstringspaces=false,
  numbers=none,
  numbers=left,
  xleftmargin=2em,
  breaklines=true,
  postbreak=\mbox{\textcolor{red}{$\hookrightarrow$}\space}
}

\renewcommand{\lstlistingname}{Listado}
\captionsetup[lstlisting]{font={footnotesize},labelfont=bf,position=bottom}
\captionsetup[figure]{font={footnotesize},labelfont=bf}
\lstnewenvironment{listadojava}[2]
{
  \lstset{language=Java,caption={#1},label={#2}}
  \noindent\minipage{\linewidth}%
}
{\endminipage}

% LISTADO SHELL
\lstnewenvironment{listadoshell}[2]
{
  \lstset{language=bash,caption={#1},label={#2}}
  \noindent\minipage{\linewidth}%
}
{\endminipage}

% LISTADO TXT
\lstnewenvironment{listadotxt}[2]
{
  \lstset{caption={#1},label={#2}, keywords={}}
  \noindent\minipage{\linewidth}%
}
{\endminipage}

% LISTADO SQL
\lstnewenvironment{listadosql}[2]%
{%
  %(el 1 )caption es #1\\%
  %(el 2) label es #2\\%
  \lstset{language=sql,caption={#1},label={#2}}%
  \noindent\minipage{\linewidth}%
}
{\endminipage}
  






\newcommand{\graficosvg}[3][14cm]{
\begin{figure}[htbp]
\centering
\textattachfile{#2.svg}{
\color{black}
\includesvg[width=#1]{#2}
}
\caption{#3}
\end{figure}
}


\newcommand{\graficosvguml}[3][7cm]{
  \texttt{\graficosvg[#1]{#2}{#3}}
}


\newcommand{\primerapagina}{
\newpage
\ficheroautoref
\tableofcontents
\newpage
}

% CAJAS
\usepackage[skins]{tcolorbox}
\newtcolorbox{Aviso}[1][Aviso]{
  enhanced,
  colback=gray!5!white,
  colframe=gray!75!black,fonttitle=\bfseries,
  colbacktitle=gray!85!black,
  attach boxed title to top left={yshift=-2mm,xshift=2mm},
  title=#1
}


\newcommand{\StudentData}{
  \begin{cuadrito}[1\textwidth]
    \vspace{0.3cm}
    \large{
      \textbf{Apellidos:} \hrulefill \\
      \textbf{Nombre:} \hrulefill \\
      \textbf{Fecha:} \hrulefill \hspace{1cm} \textbf{Grupo:} \hrulefill \\
    }
    %\vspace{0.2cm}
  \end{cuadrito}
}

% TEXTO EN MONOESPACIO PERO QUE SE PARTE EN LINEAS
\usepackage{seqsplit}
\newcommand{\tecnico}[1]{\texttt{\seqsplit{#1}}}

% REEMPLAZAR TEXTO, NO LO USO
\def\replace#1#2#3{%
 \def\tmp##1#2{##1#3\tmp}%
   \tmp#1\stopreplace#2\stopreplace}
\def\stopreplace#1\stopreplace{}

\usepackage{eurosym}


\renewcommand{\hmwkClass}{Automatización de tareas}
\renewcommand{\hmwkTitle}{Examen de \textit{scripts} de \textit{shell}}


\usepackage{blindtext}



\begin{document}

% \maketitle

% ----------------------------------------------------------------------------------------
%	TABLE OF CONTENTS
% ----------------------------------------------------------------------------------------

% \setcounter{tocdepth}{1} % Uncomment this line if you don't want subsections listed in the ToC

\primerapagina

\setlength{\parindent}{0em}
\setlength{\parskip}{1em}

\section{Normas del examen}
Es difícil evaluar el manejo de \textit{scripts} sin realizar un examen en el ordenador, pero también es difícil condensar un examen en solo dos horas. Por eso, el examen se plantea como una práctica, que el profesor corregirá en clase.

El desarrollo de este ejercicio será como el de otras prácticas. La única diferencia está en que la nota de esta práctica se tendrá en cuenta en el apartado \textit{exámenes} en vez de en el aparatado \textit{prácticas} al calcular la calificación del trimestre.

La última versión de este documento se puede descargar de \enlace{https://alvarogonzalezsotillo.github.io/apuntes-clase/sistemas-gestores-bbdd-asir2/apuntes/3/asgbd-03-examen-scripts-shell.pdf}{https://alvarogonzalezsotillo.github.io/apuntes-clase/sistemas-gestores-bbdd-asir2/apuntes/3/asgbd-03-examen-scripts-shell.pdf}




\begin{homeworkProblem}[: \textit{Scripts} de inicio y parada de \textbf{Oracle} (1 punto)]

  Crea dos \textit{scripts} para iniciar y parar \textbf{Oracle} en \path{/home/alumno/scripts/oraclestart.sh} y \path{/home/alumno/scripts/oraclestop.sh}
  
\end{homeworkProblem}

\begin{homeworkProblem}[: Arrancar automáticamente \textbf{Oracle} cuando se inicie el servidor (2 puntos)]

  \begin{itemize}
  \item \textbf{Oracle} debe levantarse cuando la máquina se inicie, y apagarse cuando la máquina se cierre.
  \item Oracle se iniciará solo si se indica en el fichero \texttt{/etc/oratab}. {\scriptsize Nota: Los \textit{scripts} no cambian el fichero \texttt{/etc/oratab}, solo lo consultan.}
  
  \item En el fichero \texttt{/home/alumno/logs/oracle.log} se dejará una traza de cuando se arrancó y se paró la máquina, y si fue necesario arrancar o parar \textit{Oracle}. Por ejemplo:
    
    \begin{listadotxt}{Ejemplo de \texttt{/home/alumno/logs/oracle.log} cuando \textbf{Oracle} se arranca}{lst:yoracle.log}
      2017-02-10-12:40:00 - Solicitud de arrancar Oracle
      2017-02-10-12:40:01 - Oracle arrancando porque /etc/oratab indica Y
      2017-02-10-12:40:20 - Oracle arrancado
    \end{listadotxt}

    \begin{listadotxt}{Ejemplo de \texttt{/home/alumno/logs/oracle.log} cuando \textbf{Oracle} se para}{lst:pararoracle}
      2017-02-10-12:41:00 - Solicitud de parar Oracle
      2017-02-10-12:41:20 - Oracle parado
    \end{listadotxt}

    \begin{listadotxt}{Ejemplo de \texttt{/home/alumno/logs/oracle.log} cuando \textbf{Oracle} no se arranca}{lst:noracle.log}
      2017-02-10-12:40:00 - Solicitud de arrancar Oracle
      2017-02-10-12:40:01 - Oracle no se arranca porque /etc/oratab indica N
    \end{listadotxt}

    
  \end{itemize}
  
\end{homeworkProblem}

\begin{homeworkProblem}[: Crea usuarios de base de datos (2 puntos)]
  Crea un script de nombre \texttt{/home/alumno/scripts/nuevo-usuario-oracle.sh} que cree un nuevo usuario de oracle. Si se invoca sin parámetros, o con más de dos, mostrará el texto de ayuda del listado \ref{lst:nuevo-usuario-ayuda}

  \begin{listadotxt}{Ayuda del script \texttt{nuevo-usuario-oracle.sh}}{lst:nuevo-usuario-ayuda}
  Crea un usuario nuevo de oracle, con permisos connect y resource.
  Si el usuario ya existe, lo desbloquea y le cambia la contraseña.
  
  Uso: nuevo-usuario-oracle.sh <usuario> <contraseña>
\end{listadotxt}


  En la salida del \textit{script} debe quedar claro si el usuario se crea (porque no existe), o solo es desbloqueado.


  
\end{homeworkProblem}


\begin{homeworkProblem}[: Almacena información periódicamente en la base de datos (3 puntos)]
  % df -k | awk '{print $3}'

  % while read -r line
  % do
  %     name="$line"
  %     echo "Name read from file - $name"
  % done < "$filename"

  Programa un \textit{script} para que cada minuto almacene en la tabla \texttt{DF} la información del comando \texttt{df -k}. Esta tabla (listado \ref{lst:tabla.sql}) tendrá como columnas:
  \begin{itemize}
  \item \texttt{hora}: Hora de lanzamiento del comando
  \item \texttt{sistema}: Nombre del tipo de sistema de ficheros
  \item \texttt{tamano}: Tamaño en KB del sistema de ficheros
  \item \texttt{usado}: Tamaño usado, en KB
  \item \texttt{montado}: Punto de montaje
  \end{itemize}

  \begin{listadoshell}{Creación de la tabla \texttt{DF}}{lst:tabla.sql}
  create table DF(
    hora varchar(40),
    sistema varchar(40),
    tamano varchar(40),
    usado varchar(40),
    montado varchar(40)
  );
  \end{listadoshell}



  \begin{listadotxt}{Ejemplo de salida del comando \texttt{df -k}}{lst:df-k}
Filesystem     1K-blocks      Used Available Use% Mounted on
udev             4002180         0   4002180   0% /dev
tmpfs             804488     19756    784732   3% /run
/dev/sda1      237874840 183034916  42733532  82% /
tmpfs            4022440    437328   3585112  11% /dev/shm
tmpfs               5120         4      5116   1% /run/lock
tmpfs            4022440         0   4022440   0% /sys/fs/cgroup
/dev/sdb5      689521880 595546232  58926896  91% /home/windows
cgmfs                100         0       100   0% /run/cgmanager/fs
tmpfs             804488        88    804400   1% /run/user/1000
  \end{listadotxt}
  

  Pistas para realizar el \textit{script}:
  \begin{itemize}
    
  \item Los \textit{heredocs} pueden contener variables: \enlace{http://superuser.com/questions/456615/how-to-pass-variables-to-a-heredoc-in-bash}{http://superuser.com/questions/456615/how-to-pass-variables-to-a-heredoc-in-bash}
    
  \item Cortar columnas con \texttt{awk}: \enlace{https://www.cyberciti.biz/tips/processing-the-delimited-files-using-cut-and-awk.html}{https://www.cyberciti.biz/tips/processing-the-delimited-files-using-cut-and-awk.html}
    
  \item Leer líneas una por una y meterlas en una variable: \enlace{http://stackoverflow.com/questions/10929453/read-a-file-line-by-line-assigning-the-value-to-a-variable}{http://stackoverflow.com/questions/10929453/read-a-file-line-by-line-assigning-the-value-to-a-variable}
    
  \item Quitar la primera línea de la salida de \texttt{df -k} con el comando \texttt{tail}: \enlace{https://linuxconfig.org/commands-on-how-to-delete-a-first-line-from-a-text-file-using-bash-shell}{https://linuxconfig.org/commands-on-how-to-delete-a-first-line-from-a-text-file-using-bash-shell}
    
  \item El \textit{script} debería seguir los siguientes pasos:
    \begin{enumerate}
    \item Quitar la primera línea de la salida de \texttt{df -k} con \texttt{tail -n +2}
      
    \item Leer cada línea con \texttt{while} y \texttt{read}
    \item Sacar los campos de cada línea con \texttt{awk}
      
    \item Ejecutar una sentencia \texttt{SQL} con los datos extraidos
    \end{enumerate}
  \end{itemize}

  
  
\end{homeworkProblem}


\begin{homeworkProblem}[: Envía un correo periódicamente (3 puntos)]

  \begin{itemize}
  \item Programa un \textit{script} para que cada minuto envíe un correo con la información promedio del comando  \texttt{df -k}. Puedes usar como base para la consulta el listado \ref{lst:promedio.sql}.
  \item El correo se enviará a \path{alvarogonzalez.profesor@gmail.com} 
  \item Con copia a \path{alvaro@alvarogonzalez.no-ip.biz}
  \item Indica tu nombre en el asunto del correo  
  \item El fichero tendrá un formato \enlace{https://stackoverflow.com/questions/643137/how-do-i-spool-to-a-csv-formatted-file-using-sqlplus\#654723}{CSV}. Se debe poder abrir directamente con excel y visualizar su resultado en filas y columnas.


\begin{listadoshell}{Consulta tipo para extraer información promedio}{lst:promedio.sql}
  select 
    sistema, avg(tamano), avg(usado), montado
  from 
    DF
  group by
    sistema, montado;
\end{listadoshell}
    
\end{itemize}
\end{homeworkProblem}


\section{Instrucciones de entrega}
\begin{itemize}
\item El ejercicio se realizará y entregará de manera individual.
\item El profesor comprobará el funcionamiento del sistema, hasta el dia 29 de Marzo.
\item No importa que el correo llegue a \path{alvarogonzalez.profesor@gmail.com}, lo que importa es que se envíe (el profesor mirará los ficheros de traza para comprobarlo)
  
\end{itemize}
\end{document}
