% Created 2022-01-21 vie 12:39
% Intended LaTeX compiler: pdflatex
\documentclass{article}
\usepackage[utf8]{inputenc}
\usepackage[T1]{fontenc}
\usepackage{graphicx}
\usepackage{longtable}
\usepackage{wrapfig}
\usepackage{rotating}
\usepackage[normalem]{ulem}
\usepackage{amsmath}
\usepackage{amssymb}
\usepackage{capt-of}
\usepackage{hyperref}
\usepackage[spanish]{babel}
\usepackage[usenames,dvipsnames]{color} % Required for custom colors
\renewcommand{\ttdefault}{pcr} % MONOESPACIO CON NEGRIT
\usepackage{lastpage}
\usepackage{listings}
\usepackage{listingsutf8}
\renewcommand{\lstlistingname}{Listado}
\lstset{frame=single,inputencoding=utf8,basicstyle=\scriptsize\ttfamily,showstringspaces=false,numbers=none}
\definecolor{MyDarkGreen}{rgb}{0.0,0.4,0.0} % This is the color used for comments
\lstset{ breaklines=true, postbreak=\mbox{\textcolor{red}{$\hookrightarrow$}\space}, keywordstyle=\bfseries, keywordstyle=[1]\color{Blue}\bfseries,  keywordstyle=[2]\color{Purple}\bfseries,  keywordstyle=[3]\color{Blue}\underbar,   identifierstyle=,   commentstyle=\usefont{T1}{pcr}{m}{sl}\color{MyDarkGreen}\small,   stringstyle=\color{Purple},   showstringspaces=false,   tabsize=2,   morecomment=[l][\color{Blue}]{...} }
\lstset{literate=  {á}{{\'a}}1 {é}{{\'e}}1 {í}{{\'i}}1 {ó}{{\'o}}1 {ú}{{\'u}}1   {Á}{{\'A}}1 {É}{{\'E}}1 {Í}{{\'I}}1 {Ó}{{\'O}}1 {Ú}{{\'U}}1   {à}{{\`a}}1 {è}{{\`e}}1 {ì}{{\`i}}1 {ò}{{\`o}}1 {ù}{{\`u}}1   {À}{{\`A}}1 {È}{{\'E}}1 {Ì}{{\`I}}1 {Ò}{{\`O}}1 {Ù}{{\`U}}1   {ä}{{\"a}}1 {ë}{{\"e}}1 {ï}{{\"i}}1 {ö}{{\"o}}1 {ü}{{\"u}}1   {Ä}{{\"A}}1 {Ë}{{\"E}}1 {Ï}{{\"I}}1 {Ö}{{\"O}}1 {Ü}{{\"U}}1   {â}{{\^a}}1 {ê}{{\^e}}1 {î}{{\^i}}1 {ô}{{\^o}}1 {û}{{\^u}}1   {Â}{{\^A}}1 {Ê}{{\^E}}1 {Î}{{\^I}}1 {Ô}{{\^O}}1 {Û}{{\^U}}1   {œ}{{\oe}}1 {Œ}{{\OE}}1 {æ}{{\ae}}1 {Æ}{{\AE}}1 {ß}{{\ss}}1   {ű}{{\H{u}}}1 {Ű}{{\H{U}}}1 {ő}{{\H{o}}}1 {Ő}{{\H{O}}}1   {ç}{{\c c}}1 {Ç}{{\c C}}1 {ø}{{\o}}1 {å}{{\r a}}1 {Å}{{\r A}}1   {€}{{\euro}}1 {£}{{\pounds}}1 {«}{{\guillemotleft}}1   {»}{{\guillemotright}}1 {ñ}{{\~n}}1 {Ñ}{{\~N}}1 {¿}{{?`}}1 }
\usepackage{caption}
\usepackage{attachfile}
\usepackage[margin=2.5cm,includeheadfoot,includehead,includefoot]{geometry}
\hypersetup{colorlinks,linkcolor=black}
\usepackage{fancyhdr}
\pagestyle{fancyplain}
\chead{}
\lhead{}
\rhead{}
\cfoot{}
\lfoot{\begin{footnotesize}alvaro.gonzalezsotillo@educa.madrid.org\end{footnotesize}}
\rfoot{\begin{footnotesize}\thepage / \pageref{LastPage}\end{footnotesize}}
\usepackage[skins]{tcolorbox}
\usepackage{multicol}
\usepackage{changepage} %ajdustwidth
\usepackage{fancybox}
\usepackage{attachfile2}
\lhead{Extraordinaria 2021 (es el \\lhead)}
\rhead{Administración y Gestión de Bases de Datos (es el \\rhead)}
\lhead{Trabajo de scripts de shell}
\rhead{ASGBD}
\usepackage{tabularx}
\usepackage{svg}
\usepackage{letltxmacro}
\LetLtxMacro{\originalincludegraphics}{\includegraphics}
\renewcommand{\includegraphics}[2][]{\IfFileExists{#2.pdf}{\originalincludegraphics[#1]{#2.pdf}}{\originalincludegraphics[#1]{#2}}}
\LetLtxMacro{\originalincludesvg}{\includesvg}
\renewcommand{\includesvg}[2][]{\IfFileExists{#2.pdf}{\originalincludegraphics[#1]{#2.pdf}}{\originalincludegraphics[#1]{#2.svg.pdf}}}
\usepackage{comment}
\excludecomment{NOTES}
\date{}
\title{}
\hypersetup{
 pdfauthor={Álvaro González Sotillo},
 pdftitle={},
 pdfkeywords={},
 pdfsubject={},
 pdfcreator={Emacs 29.0.50 (Org mode 9.4.6)}, 
 pdflang={Spanish}}
\begin{document}

\captionsetup{font=scriptsize}

\setlength{\parindent}{0em}
\setlength{\parskip}{1em}

\newtcolorbox{Aviso}[1][Aviso]{
  enhanced,
  colback=gray!5!white,
  colframe=gray!75!black,fonttitle=\bfseries,
  colbacktitle=gray!85!black,
  attach boxed title to top left={yshift=-2mm,xshift=2mm},
  title=#1
}

\newtcolorbox{cuadrito}[1][Ignorado]{
  %drop shadow southeast,
  enhanced jigsaw,
  colback=white,
}


\newcommand{\StudentData}{
  \begin{cuadrito}[1\textwidth]
    \vspace{0.3cm}
    \large{
      \textbf{Apellidos:} \hrulefill \\
      \textbf{Nombre:} \hrulefill \\
      \textbf{Fecha:} \hrulefill \hspace{1cm} \textbf{Usuario:} \hrulefill \\
    }
    \vspace{-0.2cm}
  \end{cuadrito}
}

\StudentData




La última versión de este documento se puede descargar de \url{https://alvarogonzalezsotillo.github.io/apuntes-clase/sistemas-gestores-bbdd-asir2/apuntes/3/asgbd-03-trabajo-scripts-2022.pdf}




\section{\emph{Scripts} de inicio y parada de \textbf{Oracle} (1 punto)}
\label{sec:org0000000}

Crea dos \emph{scripts} para iniciar y parar \textbf{Oracle}
\begin{itemize}
\item \texttt{/home/alumno/scripts/oraclestart.sh}
\item \texttt{/home/alumno/scripts/oraclestop.sh}
\end{itemize}


\section{Arrancar automáticamente \textbf{Oracle} cuando se inicie el servidor (2 puntos)}
\label{sec:org0000003}


\begin{itemize}
\item \textbf{Oracle} debe levantarse cuando la máquina se inicie, y apagarse cuando la máquina se cierre.
\item Oracle se iniciará solo si se indica en el fichero \texttt{/etc/oratab}.

\item En el fichero \texttt{/home/alumno/logs/oracle.log} se dejará una traza de cuando se arrancó y se paró la máquina, y si fue necesario arrancar o parar \textbf{Oracle}. Por ejemplo:

\lstset{language=bash,label= ,caption={Ejemplo de \texttt{/home/alumno/logs/oracle.log} cuando \textbf{Oracle} se arranca},captionpos=b,numbers=none}
\begin{lstlisting}
      2022-02-10-12:40:00 - Solicitud de arrancar Oracle
      2022-02-10-12:40:01 - Oracle arrancando porque /etc/oratab indica Y
      2022-02-10-12:40:20 - Oracle arrancado
\end{lstlisting}

\lstset{language=bash,label= ,caption={Ejemplo de \texttt{/home/alumno/logs/oracle.log} cuando \textbf{Oracle} se para},captionpos=b,numbers=none}
\begin{lstlisting}
      2022-02-10-12:41:00 - Solicitud de parar Oracle
      2022-02-10-12:41:20 - Oracle parado
\end{lstlisting}

\lstset{language=bash,label= ,caption={Ejemplo de \texttt{/home/alumno/logs/oracle.log} cuando \textbf{Oracle} no se arranca},captionpos=b,numbers=none}
\begin{lstlisting}
      2022-02-10-12:40:00 - Solicitud de arrancar Oracle
      2022-02-10-12:40:01 - Oracle no se arranca porque /etc/oratab indica N
\end{lstlisting}
\end{itemize}


\begin{Aviso}
\begin{itemize}
\item Los \emph{scripts} no cambian el fichero \texttt{/etc/oratab}, solo lo consultan.
\item Asegúrate de que el \emph{script} no lee líneas de comentarios, o de otros \texttt{SID}
\end{itemize}
\end{Aviso}


\section{Crea usuarios de base de datos (2 puntos)}
\label{sec:org0000006}
\begin{itemize}
\item Crea un script de nombre \texttt{/home/alumno/scripts/nuevos-usuarios-oracle.sh} que lea de su entrada estándar
\item Cada línea será un nombre de usuario a crear
\item Cada usuario se creará con permisos connect y resource
\item Si el usuario ya existe, no se hará nada con él
\end{itemize}

\lstset{language=bash,label=lst:nuevos-usuarios-ejemplo,caption={Ejemplo de uso},captionpos=b,numbers=none}
\begin{lstlisting}
  echo pepe juan manolo | ./nuevos-usuarios-oracle.sh
  pepe: no existe, se crea con contraseña pepe y se le dan roles
  juan: ya existe, no se hace nada
  manolo: no existe, se crea con contraseña manolo y se le dan roles
\end{lstlisting}

\section{Borra un usuario de la base de datos (2 puntos)}
\label{sec:org0000009}
\begin{itemize}
\item Crea un script de nombre \texttt{/home/alumno/scripts/quita-usuario-oracle.sh} que bloquee el usuario pasado como parámetro
\begin{itemize}
\item Si el usuario no existe, informará de ello y tendrá un errorlevel de \texttt{1}
\item Si se invoca con primer parámetro \texttt{-{}-drop} y segundo el usuario, borrará el usuario.
\item Si se invoca sin parámetros, o incorrectos, mostrará el texto de ayuda del listado \ref{lst:quita-usuario-ayuda}
\end{itemize}
\end{itemize}

\lstset{language=bash,label=lst:quita-usuario-ayuda,caption={Ayuda del script \texttt{quita-usuario-oracle.sh}},captionpos=b,numbers=none}
\begin{lstlisting}

  Borra o bloquea un usuario de oracle.
  
  Uso: quita-usuario-oracle.sh <usuario>
       quita-usuario-oracle.sh --drop <usuario>
\end{lstlisting}

\begin{Aviso}
En la salida del \emph{script} debe quedar claro si el usuario se borra o bloquea
\end{Aviso}



\section{Almacena información periódicamente en la base de datos (3 puntos)}
\label{sec:org000000c}
Programa un \emph{script} para que cada minuto almacene en la tabla \texttt{ESTADO} la siguiente información

\begin{itemize}
\item \texttt{uptime}: Tiempo activo del servidor (\texttt{uptime -p})
\item \texttt{procesos}: Número de procesos activos en el servidor
\item \texttt{disco}: Número de KB usados en el punto de montaje raíz
\item \texttt{conexiones}: Número de conexiones TCP abiertas
\end{itemize}


\lstset{language=SQL,label=lst:tabla.sql,caption={Creación de la tabla \texttt{ESTADO}},captionpos=b,numbers=none}
\begin{lstlisting}
  create table ESTADO(
    uptime varchar(40),
    procesos varchar(40),
    disco varchar(40),
    conexiones varchar(40)
  );
\end{lstlisting}


\begin{Aviso}
Pistas para realizar el \emph{script}:
\begin{itemize}
\item Los \emph{heredocs} pueden \href{http://superuser.com/questions/456615/how-to-pass-variables-to-a-heredoc-in-bash}{contener variables}
\item \href{https://www.cyberciti.biz/tips/processing-the-delimited-files-using-cut-and-awk.html}{Cortar columnas} con \texttt{awk}
\item \href{https://linuxconfig.org/commands-on-how-to-delete-a-first-line-from-a-text-file-using-bash-shell}{Quitar las primeras líneas} de la salida con el comando \texttt{tail}
\end{itemize}
\end{Aviso}


\section{Envía un correo periódicamente (2 puntos)}
\label{sec:org000000f}
\begin{itemize}
\item Programa un \emph{script} para que cada minuto envíe un correo con la información de la tabla \texttt{ESTADO}.
\item El correo se enviará a \texttt{alvarogonzalez.profesor@gmail.com}
\item Con copia a \texttt{alvaro@alvarogonzalez.no-ip.biz}
\item Indica tu nombre en el asunto del correo
\item El fichero tendrá \href{https://stackoverflow.com/questions/643137/how-do-i-spool-to-a-csv-formatted-file-using-sqlplus\#654723}{formato CSV}. Se debe poder abrir directamente con \textbf{excel} y visualizar su resultado en filas y columnas.
\end{itemize}





\section{Instrucciones de entrega}
\label{sec:org0000012}

\begin{itemize}
\item El ejercicio se realizará y entregará de manera individual. Solo se puede hacer por parejas si no hay ordenadores suficientes.
\item El profesor comprobará el funcionamiento del sistema el día indicado.
\item Sube en la tarea del aula virtual un ZIP con todos los ficheros que has creado o modificado:
\begin{itemize}
\item \emph{Scripts}
\item \emph{units} de \texttt{systemd}
\item Ficheros de \texttt{cron} / \texttt{anacron}
\end{itemize}
\end{itemize}
\end{document}
